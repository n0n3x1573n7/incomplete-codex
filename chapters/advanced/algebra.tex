%\documentclass{report}

\begin{document}
    \section{Group Basics}
    
    The first thing we would encounter in abstract algebra is a group... but you already encountered it. Refer to the chapter "Algebraic Structures" for the definition of a group and an abelian group.
    
    If the context is obvious, we will skip $\cdot$ and write $ab$ instead of $a \cdot b$. The identity of $G$ will be denoted $e$ or $1$.
    
    \begin{defn} \label{def_group_power}
        The product of $n$ occurrences of $x$ is denoted $x^n$. The product of $n$ occurrences of $x^{-1}$ is denoted $x^{-n}$. Also $x^0 = 1$.
    \end{defn}
    
    \begin{thm} \label{thm_group_basics}
        If $G$ is a group, and $a,b,c \in G$, then \begin{enumerate}
            \item the identity of $G$ is unique.
            \item the inverse $a^{-1}$ is unique.
            \item $(a^{-1})^{-1}=a$.
            \item $(ab)^{-1} = b^{-1}a^{-1}$.
            \item if $ab=ac$, then $b=c$. Also, if $ba=ca$, then $b=c$.
            \item For $n,m \in \mathbb{Z}$, $x^nx^m = x^{n+m}$ and $(x^n)^{-1} = x^{-n}.$
        \end{enumerate}
    \end{thm}
    
    \begin{proof}
        \begin{enumerate} \item[]
        \item Let $e_1$ and $e_2$ be the identities of $G$. Then $e_1e_2=e_2e_1=e_1$, and $e_2e_1=e_1e_2=e_2$, from the definition of the identity. Therefore $e_1=e_2$.
        \item Let $b_1$ and $b_2$ the inverses of $a$. Then $b_1 = b_1(ab_2) = (b_1a)b_2 = b_2$.
        \item The definition of an inverse shows that $a$ is an inverse of $a^{-1}$. From (ii), such an inverse is unique.
        \item $(ab)b^{-1}a^{-1} = a(bb^{-1})a^{-1} = aa^{-1} = e$. Similarly $b^{-1}a^{-1}(ab) = e$. From (ii), the inverse of $ab$ is unique.
        \item $ab=ac \implies a^{-1}ab=a^{-1}ac \implies b=c$. Similar argument for $ba=ca$.
        \item TODO
        \end{enumerate}
    \end{proof}
    
    \begin{defn} \label{def_group_misc}
        Let $G$ be a group. \begin{itemize}
        \item $a,b \in G$ \emph{commute} if $ab=ba$.
        \item The \emph{order} of $x \in G$, denoted $|x|$, is the smallest positive integer $n$ such that $x^n=1$. If no such $n$ exists, then $|x|=\infty$.
        \item TODO: subgroup
        \item TODO: cyclic subgroup
    \end{itemize} \end{defn}
    
    \begin{defn} \label{def_group_example} \begin{itemize}
        \item[]
        \item TODO: Z/nZ
        \item TODO: Sn
    \end{itemize} \end{defn}
    
    \begin{defn}[Homomorphisms and Isomorphisms] \label{def_group_homomorphism}
        Let $(G,\star)$, $(H,\diamond)$ be two groups. Then a map $\varphi: G \to H$ is a \emph{homomorphism} if for all $x,y \in G$, $\varphi(x \star y)=\varphi(x) \diamond \varphi(y)$. An \emph{isomorphism} is a bijective homomorphism.
    \end{defn}

\end{document}