%\documentclass{report}

\begin{document}
    \section{Metric Spaces}
    
    \subsection{Topology of Metric Spaces}
    
    \begin{defn}[Metric Space] \label{def_metric_space}
        A set $X$ equipped with a function $d:X \times X \rightarrow \mathbb{R}$ is a \emph{metric space} if $d$ satisfies, for all $p,q,r \in X$:\begin{enumerate}
            \item $d(p,q)>0$ for $p \neq q$, and $d(p,p)=0$.
            \item $d(p,q)=d(q,p)$.
            \item $d(p,q) \leq d(p,r)+d(r,q)$. This inequality is called the \emph{triangle inequality}.
        \end{enumerate}
        The elements of $X$ are called \emph{point}s. The function $d$ is called a \emph{metric}.
    \end{defn}
    
    \begin{defn} \label{def_analysis_top}
        Let $X$ be a metric space, $E \subseteq X$, and $p \in X$. \begin{itemize}
            \item A \emph{neighborhood} of $p$, denoted $N_r(p)$, is $\{q \in X | d(p,q) \leq r \}$, where $r>0$.
            \item $p$ is a \emph{limit point} of $E$ if every neighborhood of $p$ contains $q \in E$ different from $p$. The set of all limit points of $E$ is denoted $E'$.
            \item The \emph{boundary} of $E$ is (TODO)
            \item $p$ is an \emph{interior point} of $E$ if there is a neighborhood of $p$ that is contained in $E$.
            \item $p$ is an \emph{isolated point} of $E$ if $p \in E$ and $p$ is not a limit point of $E$.
            \item $E$ is \emph{open} if every point in $E$ is an interior point.
            \item $E$ is \emph{closed} if if every limit point of $E$ is in $E$.
            \item $E$ is \emph{bounded} if there is a neighborhood of some $p$ that contains $E$.
            \item $E$ is \emph{dense} if every point of $X$ is a limit point of $E$ or a point of $E$.
        \end{itemize}
    \end{defn}
    
    Here is a figure demonstrating these notions in the space $\mathbb{R}$ with the metric $d(x,y) = |x-y|$ and $E = [0,1) \cap \{2\}$:
    
    (TODO)
    
    Note that a set can be both open and closed. For example, an empty set is (vacuously) both open and closed. $X$ itself is also both open and closed. The notions in topology will be covered in greater detail in the Topology chapter.
    
    \textbf{IMPORTANT: From now on in this chapter, assume $X$ is always a metric space with the metric $d$, and $E \subseteq X$, unless stated otherwise.}
    
    \begin{prop} \label{prop_analysis_top} \begin{enumerate} \item[]
        \item A neighborhood is open.
        \item If $p$ is a limit point of $E$, then every neighborhood contains infinitely many points of $E$.
        \item $E$ is open iff $E^C$ is closed.
        \item $E$ is closed iff $E^C$ is open.
    \end{enumerate} \end{prop}
    
    \begin{proof}
    1. Let $q \in N_r(p)$. Then $N_{r-d(p,q)}(q) \subseteq N_r(p)$ because, if $x \in N_{r-d(p,q)}(q)$, then $d(p,x) \leq d(p,q)+d(q,r) < d(p,q)+r-d(p,q) = r$ so $x \in N_r(p)$.
    
    2. Suppose some neighborhood $N_r(p)$ contains only finitely many points of $E$, namely $x_1$, $\cdots$, $x_k$. Let $r=min_{i=1}^k d(p,x_i)$. Then $N_r(p)$ contains no points of $E$, contradiction.
    
    3. Suppose $E$ is open and $x$ is a limit point of $E^C$. Then since every neighborhood of $x$ intersects $E^C$, $x$ is not an interior point of $E$. Therefore $x \in E^C$. Conversely, suppose $E^C$ is closed and $x \in E$. Since $x \notin E^C$, $x$ is not a limit point of $E^C$. Therefore there is a neighborhood of $x$ which does not intersect $E^C$, and that is contained in $E$. Therefore $x$ is an interior point. 
    
    4. $E = (E^C)^C$.
    \end{proof}
    
    \begin{prop} \label{prop_analysis_closure} \begin{enumerate} \item[]
        \item TODO
    \end{enumerate} \end{prop}
    
    \begin{proof}
        TODO
    \end{proof}
    
    \subsection{Compact Sets}
    
    \begin{defn}[Compact Set] \label{def_analysis_compact}
        An \emph{open cover} of $E$ is a collection of open subsets of $X$ whose union contains $E$. A \emph{finite subcover} of an open cover is a finite subset whose union still contains $E$. $E$ is \emph{compact} if every open cover of $E$ contains a finite subcover.
    \end{defn}
    
    TODO
    
    \section{Sequences}
    
    \begin{defn}[Convergence] \label{def_analysis_converge}
        A sequence $\{p_n\}$ in $X$ \emph{converges} to $p \in X$ if, for every $\epsilon > 0$, there is an integer $N$ such that $n \geq N$ implies $d(p_n, p) < \epsilon$. We also write $p_n \rightarrow p$, or $lim_{n \rightarrow p} p_n = p$. A sequence \emph{diverges} if it does not converge.
    \end{defn}
    
    \begin{prop} \label{prop_analysis_converge}
        Let $\{p_n\}$ be a sequence in $X$. \begin{enumerate}
        \item $\{p_n\} \rightarrow p \in X$ iff for every $N_r(p)$, there are only finitely many terms of $\{p_n\}$ that are not in $N_r(p)$.
        \item If $\{p_n\}$ converges to both $p,q \in X$, then $p = q$.
        \item If $\{p_n\}$ converges, then it is bounded.
        \item If $p$ is a limit point of $E$, then there is a sequence in $E$ that converges to $p$.
    \end{enumerate} \end{prop}
    
    \begin{proof}
        1. Suppose $\{p_n\} \rightarrow p \in X$. Then for every $r > 0$, there is $N$ such that $n \geq N$ implies $d(p_n, p) < r$, i.e. $p_n \in N_r(p)$. Conversely, given $\epsilon > 0$, suppose there are only finitely many terms $p_{n_1}, p_{n_2}, \cdots, p_{n_k}$ that are not in $N_\epsilon (p)$. Then $n \geq n_k+1$ implies $p_n \in N_\epsilon (p)$, i.e. $d(p_n, p) < \epsilon$.
        
        2. Given any $\epsilon > 0$, take $N, M$ such that $n \geq N$ implies $d(p_n, p) < \epsilon / 2$ and $n \geq M$ implies $d(p_n, q) < \epsilon / 2$. Then $n \geq max(N,M)$ implies $d(p,q) \leq d(p_n,p) + d(p_n,q) < \epsilon$. Since $\epsilon$ is arbitrary, $p=q$.
        
        3. Let $p_n \rightarrow p$. Take $N$ such that $n \geq N$ implies $d(p_n, p) < 1$. Then every $p_n$ satisfies $d(p_n, p) \leq max(1, d(p_1,p), \cdots, d(p_N,p))$.
        
        4. Take each $p_n$ as any point in $E \cap N_{1/n}(p)$. Then for any $\epsilon > 0$, there is $N > 1/\epsilon$, and $n > N$ implies $d(p_n, p) < \epsilon$. Therefore $p_n \rightarrow p$.
    \end{proof}
    
    \begin{defn}[Cauchy Sequence] \label{def_analysis_cauchy}
        A sequence $\{p_n\}$ in $X$ is \emph{Cauchy} if for every $\epsilon > 0$ there is an integer $N$ such that $n,m \geq N$ implies $d(p_n,p_m) < \epsilon$.
    \end{defn}
    
    Every convergent sequence is Cauchy, as we will show, but not every Cauchy sequence converges. For example, $\{1/n\}$ in the metric space $(0,1]$ does not converge.
    
    \begin{prop} \label{prop_analysis_convcauchy}
        Every convergent sequence is Cauchy.
    \end{prop}
    
    \begin{proof}
        Let $p_n \rightarrow p$. Given $\epsilon > 0$, take $N$ such that $n \geq N$ implies $d(p_n, p) < \epsilon / 2$. Then $n,m \geq N$ implies $d(p_n, p_m) \leq d(p_n,p) + d(p_m,p) < \epsilon$.
    \end{proof}
    
    \section{Series}
    
    \section{Continuity}
    
    \section{Differentiation}
    
    \section{Integral}
    
    \section{Sequences and Series of Functions}

\end{document}