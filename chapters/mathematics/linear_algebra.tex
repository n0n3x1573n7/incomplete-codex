\documentclass{report}

\begin{document}
Linear algebra, in sum, deals with vectors and matrices. It targets to solve a given set of linear equations with multiple variables.
	\section{Vector Spaces}
		For the definitions on vector spaces, subspaces, and bases, refer to the chapter \ref{chap_vector_space}.
		\subsection{Linear Independence}
		We now define linear independence, one of the most important concepts utilized in linear algebra.
		\begin{defn}[Linear Independence]
			if $S=\{\vec{v_1}, \vec{v_2}, \dots, \vec{v_r}\}$ is a nonempty set of vectors in a vector space $V$, then the vector equation $k_1\vec{v_1}+k_2\vec{v_2}+\dots+k_r\vec{v_r}=\vec{0}$ has at least one solution, namely, $k_1=0, k_2=0, \dots, k_r=0$, the \emph{trivial solution}. If this is the only solution, then $S$ is said to be a \emph{linearly independent set}. If there are solutions in addition to the trivial solution, then $S$ is said to be \emph{linearly dependent}.
		\end{defn}
		
		\begin{thm}
			Let $S=\{\vec{v_1}, \vec{v_2}, \dots, \vec{v_r}\}$ be a set of vectors in $\mathbb{R}^n$. If $r>n$, then $S$ is linearly dependent.
		\end{thm}
		
		\subsection{Orthogonality}
		\begin{defn}[Euclidean Inner Product]
			Let $\vec{u}=(u_1,u_2,\dots,u_n)$ and $\vec{v}=(v_1,v_2,\dots,v_n)$ in $\mathbb{R}^n$. The inner product of the two vectors $\vec{u}$ and $\vec{v}$ is defined as
			\begin{displaymath}
				\vec{u}\cdot\vec{v}=\sum_{i=1}^{n}u_iv_i=u_1v_1+u_2v_2+\dots+u_nv_n
			\end{displaymath}
		\end{defn}
		
		\begin{defn}[Norm]
			The \emph{norm} of $\vec{u}=(u_1,u_2,\dots,u_n)$ in $\mathbb{R}^n$, denoted $\|\vec{u}\|$, is defined by
			\begin{displaymath}
			\|\vec{u}\|=\sqrt{\vec{u} \cdot \vec{u}}=\sqrt{\sum_{i=1}^{n}u_i^2}=\sqrt{u_1^2+u_2^2+\dots+u_n^2}
			\end{displaymath}
		\end{defn}
		
		\begin{defn}[Unit vectors]
			A vector $\vec{u}$ in $\mathbb{R}^n$ is said to be a unit vector iff $\|\vec{u}\|=1$.
		\end{defn}
		
		\begin{defn}[Angle]
			The \emph{angle} between two nonzero vectors $\vec{u}$ and $\vec{v}$ in $\mathbb{R}^n$ is defined by
			\begin{displaymath}
				\theta=\cos^{-1}(\frac{\vec{u}\cdot\vec{v}}{\|\vec{u}\|\|\vec{v}\|})
			\end{displaymath}
		\end{defn}
		
		\begin{defn}[Orthogonal Vectors]
			Two nonzero vectors $\vec{u}$ and $\vec{v}$ in $\mathbb{R}^n$ are said to be \emph{orthogonal} or \emph{perpendicular} if $\vec{u} \cdot \vec{v} = 0$.
		\end{defn}
		
		\begin{defn}[Orthogonal set]
			A nonempty set of vectors in $\mathbb{R}^n$ is called an \emph{orthogonal set} if all pairs of distinct vectors in the set are orthogonal. If they are also all unit vectors, it is called an \emph{orthonormal set}.\\
			In other words, for a set $\{u_1, u_2, \dots, u_n\}$ to be orthogonal:
			\begin{displaymath}
				u_i \cdot u_j
				\begin{cases}
					\|u_i\|^2 & i=j\\
					0 & i \ne j
				\end{cases}
			\end{displaymath}
			And for the set to be orthonormal, in addition to above, $\forall i \in \{1, 2, \dots n\}, \|u_i\|=1$.
		\end{defn}
		
	\section{Solving Equations}
		\subsection{}
	\section{Determinants}
		\subsection{}
	\section{Eigenvalues and Eigenvectors}
		\subsection{}
\end{document}

%Introduction to Linear Algebra 4th ed, Gilbert Strang