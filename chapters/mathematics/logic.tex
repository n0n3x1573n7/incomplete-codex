\documentclass{report}

\begin{document}
	There wouldn't be math or any branch of science if there weren't logic. In this section, basic mathematical proofs and the methods of proof will be discussed.
	\section{Boolean Algebra}
	Most branches of mathematics use propositions; that is, mathematical statements that can be determined to be either true or false. In Boolean algebra, variables and constants can take on two values: true(1) or false(0). By taking the statements to be the variables in Boolean algebra, we can think of mathematical statements as formulas of Boolean algebra.
	
	In Boolean algebra, there are only two values, true(1) and false(0), and three basic operators, two of which are binary and one unary.
	
	AND operator(conjunction), often denoted as $p \cdot q$ or $p \wedge q$, has the value true iff p and q are both true; false if either $p$ or $q$ are false. The truth-table for the AND operator is as follows:
	
	\begin{center}
	\begin{tabular}{ccc}
		$p$ & $q$ & $p \wedge q$ \\
		0 & 0 & 0\\
		0 & 1 & 0\\
		1 & 0 & 0\\
		1 & 1 & 1
	\end{tabular}
	\end{center}
	
	OR operator(disjunction), often denoted as $p + q$ or $p \vee q$, has the value false iff $p$ and $q$ are both false; true if either $p$ or $q$ are true. The truth-table for the OR operator is as follows:
	
	\begin{center}
	\begin{tabular}{ccc}
		$p$ & $q$ & $p \vee q$ \\
		0 & 0 & 0\\
		0 & 1 & 1\\
		1 & 0 & 1\\
		1 & 1 & 1
	\end{tabular}
	\end{center}
	
	NOT operator(negation), often denoted as $p'$, $\textasciitilde p$, or $\neg p$, is a unary operator. The operator switched the state of the variable, that is, if it is true its value is false; if false the value is true. The truth-table for the NOT operator is as follows:
	
	\begin{center}
	\begin{tabular}{cc}
		$p$ & $\neg p$\\
		0 & 1\\
		1 & 0
	\end{tabular}
	\end{center}
	
	Derived by composition of the basic operators, there are many secondary operators: to name the most important operators, implication($\rightarrow$), exclusive-or(XOR, $\bigoplus$), and equivalence($=$, $\equiv$). The truth-table for the operators are as follows:
	
	\begin{center}
	\begin{tabular}{ccccc}
		$p$ & $q$ & $p \rightarrow q$ & $p \bigoplus q$ & $p \equiv q$ \\
		0   & 0   & 1                 & 0               & 1     \\
		0   & 1   & 1                 & 1               & 0     \\
		1   & 0   & 0                 & 1               & 0     \\
		1   & 1   & 1                 & 0               & 1    
	\end{tabular}
	\end{center}
	
	The operators are derived as follows:\\
	\begin{center}
	\begin{tabular}{lllll}
		$p \rightarrow q$ & $=$ & $\neg p \vee y$                       &   &                                          \\
		$p \bigoplus q$   & $=$ & $(p \vee q) \wedge \neg (p \wedge q)$ & $=$ & $(p \wedge \neg q) \vee (\neg p \wedge q)$ \\
		$p \equiv q$      & $=$ & $\neg (p \bigoplus q)$                & $=$ & $(p \wedge q) \vee (\neg p \wedge \neg q)$
	\end{tabular}
	\end{center}
	
	\section{Proof Techniques}
\end{document}