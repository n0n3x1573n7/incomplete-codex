\documentclass{report}
\renewcommand{\numberline}[1]{#1~}

\usepackage{fontspec}
\setmainfont{Courier New}%\setmainfont{D2Coding}

\usepackage[hidelinks,unicode,bookmarks=true]{hyperref}
\usepackage[usenames,dvipsnames]{color}
\usepackage[round]{natbib}
\hypersetup{colorlinks,
	citecolor=Red,
	linkcolor=Green,
	urlcolor=Blue}

\usepackage{fancyvrb}
\usepackage{graphicx}
\usepackage{subfig}
\usepackage{amsmath}
\usepackage{amsthm}
\usepackage{amssymb}
%\usepackage{relsize}
\usepackage{centernot}
\usepackage[top=3cm, left=3cm, right=3cm, bottom=2cm]{geometry}
\usepackage{titling}
%\usepackage{lipsum}
\usepackage{standalone}
\usepackage{enumitem}
\usepackage{verbatim}
\usepackage{multirow}

\usepackage{ulem}

\usepackage{bm}

\usepackage{tikz}
\newcommand*\circled[1]{\tikz[baseline=(char.base)]{\node[shape=circle,draw,inner sep=2pt] (char) {#1};}}

\newcommand{\zo}{\{0,1\}}
\newcommand{\strs}{\{0,1\}^\ast}
\newcommand{\ddx}{\frac{d}{dx}}

\setlength{\droptitle}{-3em}

\newtheoremstyle{break}% name of the style to be used
{\topsep}% measure of space to leave above the theorem
{\topsep}% measure of space to leave below the theorem
{}% name of font to use in the body of the theorem
{}% measure of space to indent
{\bfseries}% name of head font
{}% punctuation between head and body
{\newline}% space after theorem head; " " = normal interword space
{}% Manually specify head

\theoremstyle{break}
\newtheorem{defn}{Definition}
\newtheorem{axiom}[defn]{Axiom}
\newtheorem{thm}[defn]{Theorem}
\newtheorem{lemma}[defn]{Lemma}
\newtheorem{prop}[defn]{Proposition}
\newtheorem{coro}[defn]{Corollary}
\newtheorem{thesis}[defn]{Thesis}
\newtheorem{mthd}[defn]{Method}
\newtheorem*{remark}{Remark}
\newtheorem*{note}{Note}
\newtheorem*{prob}{Problem}
\newtheorem*{soln}{Solution}
\newtheorem*{claim}{Claim}
\renewcommand*{\proofname}{Proof}

\title{
	The Incomplete Codex of Mathematics for Computer Scientists\\
	\large From Programmers to Hackers: Mathematical Basis to Computer Science
}

\author{None(\href{https://www.twitter.com/n0n3x1573n7}{@n0n3x1573n7}), jh05013(\href{https://twitter.com/jh17916681}{@jh17916681})}

\date{\today}

\begin{document}
	\maketitle

	\chapter{Introduction}
		\input{./chapters/introduction.tex}
	
	\tableofcontents
	
	\part{Mathematical Preliminaries}
	
		\chapter{Logic}
		    \documentclass{report}

\begin{document}
	There wouldn't be math or any branch of science if there weren't logic. In this section, basic mathematical proofs and the methods of proof will be discussed.
	\section{Boolean Algebra}
	Most branches of mathematics use propositions; that is, mathematical statements that can be determined to be either true or false. In Boolean algebra, variables and constants can take on two values: true(1) or false(0). By taking the statements to be the variables in Boolean algebra, we can think of mathematical statements as formulas of Boolean algebra.
	
	In Boolean algebra, there are only two values, true(1) and false(0), and three basic operators, two of which are binary and one unary.
	
	AND operator(conjunction), often denoted as $p \cdot q$ or $p \wedge q$, has the value true iff p and q are both true; false if either $p$ or $q$ are false. The truth-table for the AND operator is as follows:
	
	\begin{center}
	\begin{tabular}{ccc}
		$p$ & $q$ & $p \wedge q$ \\
		0 & 0 & 0\\
		0 & 1 & 0\\
		1 & 0 & 0\\
		1 & 1 & 1
	\end{tabular}
	\end{center}
	
	OR operator(disjunction), often denoted as $p + q$ or $p \vee q$, has the value false iff $p$ and $q$ are both false; true if either $p$ or $q$ are true. The truth-table for the OR operator is as follows:
	
	\begin{center}
	\begin{tabular}{ccc}
		$p$ & $q$ & $p \vee q$ \\
		0 & 0 & 0\\
		0 & 1 & 1\\
		1 & 0 & 1\\
		1 & 1 & 1
	\end{tabular}
	\end{center}
	
	NOT operator(negation), often denoted as $p'$, $\textasciitilde p$, or $\neg p$, is a unary operator. The operator switched the state of the variable, that is, if it is true its value is false; if false the value is true. The truth-table for the NOT operator is as follows:
	
	\begin{center}
	\begin{tabular}{cc}
		$p$ & $\neg p$\\
		0 & 1\\
		1 & 0
	\end{tabular}
	\end{center}
	
	Derived by composition of the basic operators, there are many secondary operators: to name the most important operators, implication($\rightarrow$), exclusive-or(XOR, $\bigoplus$), and equivalence($=$, $\equiv$). The truth-table for the operators are as follows:
	
	\begin{center}
	\begin{tabular}{ccccc}
		$p$ & $q$ & $p \rightarrow q$ & $p \bigoplus q$ & $p \equiv q$ \\
		0   & 0   & 1                 & 0               & 1     \\
		0   & 1   & 1                 & 1               & 0     \\
		1   & 0   & 0                 & 1               & 0     \\
		1   & 1   & 1                 & 0               & 1    
	\end{tabular}
	\end{center}
	
	The operators are derived as follows:\\
	\begin{center}
	\begin{tabular}{lllll}
		$p \rightarrow q$ & $=$ & $\neg p \vee y$                       &   &                                          \\
		$p \bigoplus q$   & $=$ & $(p \vee q) \wedge \neg (p \wedge q)$ & $=$ & $(p \wedge \neg q) \vee (\neg p \wedge q)$ \\
		$p \equiv q$      & $=$ & $\neg (p \bigoplus q)$                & $=$ & $(p \wedge q) \vee (\neg p \wedge \neg q)$
	\end{tabular}
	\end{center}
	
	
	
	\section{Proof Techniques}
	There are many methods of proof. In this section, the common methods of proof used in mathematics will be discussed.
	\subsection{Direct Proof}
	
	\subsection{Proof by Mathematical Induction}
	
	\subsection{Proof by Contraposition}
	
	\subsection{Proof by Construction}
	
	\subsection{Proof by Exhaustion}
	
	\subsection{Computer-assisted Proof}
	
\end{document}

		\chapter{Algebraic Structures}
		    \documentclass{report}

\begin{document}

	\section{Algebraic Structures}

		\subsection{Sets}

			\begin{defn}[Set] \label{def_set}
				A \emph{set} is a collection of distinct objects. 
			\end{defn}
		
			To see some traits on sets, we literally start from nothing:
			\begin{axiom}[Empty Set Axiom] \label{axiom_empty_set}
				There is a set containing no members, that is:
				\begin{displaymath}
				\exists B \text{ such that } \forall x, (x \notin B)
				\end{displaymath}
				We call this set the \emph{empty set}, and denote it by the symbol $\emptyset$.
			\end{axiom}

			We now have $\emptyset$; we now write down a few rules for how to manipulate sets.
			\begin{axiom}[Axiom of Extensionality] \label{axiom_extensionality}
				Two sets are equal if and only if they share the same elements, that is:
				\begin{displaymath}
				\forall A,B [\forall z,((z \in A) \Leftrightarrow (z \in B)) \Rightarrow (A=B)]
				\end{displaymath}
			\end{axiom}

			\begin{axiom}[Axiom of Pairing] \label{axiom_pairing}
				Given any two sets $A$ and $B$, there is a set which have the members just $A$ and $B$, that is:
				\begin{displaymath}
					\forall A,B \exists C \forall x [x \in C \Leftrightarrow ((x=A) \vee (x=B))]
				\end{displaymath}
				If $A$ and $B$ are distinct sets, we write this set $C$ as $\{A,B\}$; if $A=B$, we write it as $\{A\}$.
			\end{axiom}
	
			\begin{axiom}[Axiom of Union, simple version] \label{axiom_union_simple}
				Given any two sets $A$ and $B$, there is a set whose members are those sets belonging to either $A$ or $B$, that is:
				\begin{displaymath}
					\forall A,B \exists C \forall x [x \in C \Leftrightarrow ((x \in A) \vee (x \in B))]
				\end{displaymath}
				We write this set $C$ as $A \cup B$.
			\end{axiom}
			
			In the simplified version of Axiom of Union, we take union of only two things, but we sometimes we want to take unions of more than two things or even more than finitely many things. This is given by the full version of the axiom:
		
			\begin{axiom}[Axiom of Union, full version] \label{axiom_union_full}
				Given any set $A$, there is a set $C$ whose elements are exactly the members of the members of $A$, that is:
				\begin{displaymath}
					\forall A \exists C [x \in C \Leftrightarrow (\exists A'(A' \in A) \wedge (x \in A'))]
				\end{displaymath}
				We denote this set $C$ as
				\begin{displaymath}
					\bigcup_{A' \in A}A'
				\end{displaymath}
			\end{axiom}

			\begin{axiom}[Axiom of Intersection, simple version] \label{axiom_intersection_simple}
				Given any two sets $A$ and $B$, there is a set whose members are member of both $A$ and $B$, that is:
				\begin{displaymath}
					\forall A,B \exists C \forall x [(x \in C) \Leftrightarrow ((x \in A ) \wedge (x \in B))]
				\end{displaymath}
			\end{axiom}
	
			Sometimes as union, we would want to take intersection of more than finitely many things. This is given by the full version of the axiom:
			
			\begin{axiom}[Axiom of Intersection, full version] \label{axiom_intersection_full}
				Given any set $A$, there is a set $C$ whose elements are exactly the members of all members of $A$, that is:
				\begin{displaymath}
					\forall A \exists C \forall x [(x \in C) \Leftrightarrow (\forall A'((A' \in A) \Rightarrow (x \in A')))]
				\end{displaymath}
				We denote this set $C$ as
				\begin{displaymath}
					\bigcap_{A' \in A}A'
				\end{displaymath}
			\end{axiom}
		
			\begin{axiom}[Axiom of Subset] \label{axiom_subset}
				For any two sets $A$ and $B$, we say that $B \subset A$ if and only if every member of B is a member of A, that is:
				\begin{displaymath}
					(B \subseteq A) \Leftrightarrow (\forall x (x \in B) \Rightarrow (x \in A))
				\end{displaymath}
			\end{axiom}
			
			By the Axiom of Subset we can define the power set of an any given set:
			
			\begin{defn}[Power Set] \label{def_powerset}
				For any set $A$, the \emph{power set} of the set $A$, denoted $P(A)$, whose members are precisely the collection of all possible subsets of A, that is:
				\begin{displaymath}
					\forall A \exists P(A) \forall B((B \subseteq A) \Leftrightarrow (B \in P(A)))
				\end{displaymath}
			\end{defn}
		
			\begin{defn}[Equivalence Relation] \label{def_equiv_rel}
				Let $S$ be a set. An \emph{Equivalence Relation} on $S$ is a relation, denoted by $\textasciitilde$, with the following properties, $\forall a,b,c \in S$:
				\begin{itemize}
					\item \textbf{Reflexivity} $a\textasciitilde a$
					\item \textbf{Symmetry} $a\textasciitilde b \Leftrightarrow b\textasciitilde a$
					\item \textbf{Transitivity} $(a\textasciitilde b) \wedge (b\textasciitilde c) \Rightarrow (a\textasciitilde c)$
				\end{itemize}
			\end{defn}
		
			\begin{defn}[Setoid] \label{def_setoid}
				A \emph{setoid} is a set in which an equivalence relation is defined, denoted $(S,\textasciitilde )$.
			\end{defn}
	
			\begin{defn}[Equivalence Class] \label{def_equivalence_class}
				The equivalence class of $a\in S$ under $\textasciitilde $, denoted $[a]$, is defined as $[a]=\{b \in S|a \textasciitilde b\}$.
			\end{defn}

			\begin{defn}[Order] \label{def_order}
				Let $S$ be a set. An \emph{order} on $S$ is a relation, denoted by $<$, with the following properties:
				\begin{itemize}
					\item If $x \in S$ and $y \in S$ then one and only one of the following statements is true:
					\begin{displaymath}
						x<y, x=y, y<x
					\end{displaymath}
					\item For $x,y,z \in S$, if $x<y$ and $y<z$, then $x<z$.
				\end{itemize}
			\end{defn}

			\begin{remark} \label{remark_order}
				\begin{itemize}
					\item[]
					\item It is possible to write $x>y$ in place of $y<x$
					\item The notation $x \leq y$ indicates that $x<y$ or $x=y$.
				\end{itemize}
			\end{remark}

		\begin{defn}[Ordered Set] \label{def_ordered_set}
			An \emph{ordered set} is a set in which an order is defined, denoted $(S,<)$.
		\end{defn}

		\begin{defn}[Bound] \label{def_bound}
			Suppose $S$ is an ordered set, and $E\subset S$.\\
			If there exists $\beta \in S$ such that $x \leq \beta$ for every $x \in E$, we say that E is \emph{bounded above}, and call $\beta$ an \emph{upper bound} of E.
			If there exists $\alpha \in S$ such that $x \geq \alpha$ for every $x \in E$, we say that E is \emph{bounded below}, and call $\alpha$ a \emph{lower bound} of E.
		\end{defn}

		\begin{defn}[Least Upper Bound] \label{def_supremum}
			Suppose that $S$ is an ordered set, and $E \subset S$.
			If there exists a $\beta \in S$ with the following properties:
			\begin{itemize}
				\item $\beta$ is an upper bound of $E$
				\item If $\gamma < \beta$, then $\gamma$ is not an upper bound of E
			\end{itemize}
			Then $\beta$ is called the \emph{Least Upper Bound} of E or the \emph{supremum} of E, denoted
			\begin{displaymath}
				\beta=sup(E)
			\end{displaymath}
		\end{defn}

		\begin{defn}[Greatest Lower Bound] \label{def_infimum}
			Suppose that $S$ is an ordered set, and $E \subset S$.
			If there exists a $\alpha \in S$ with the following properties:
			\begin{itemize}
				\item $\alpha$ is a lower bound of $E$
				\item If $\gamma < \alpha$, then $\gamma$ is not an lower bound of E
			\end{itemize}
			Then $\alpha$ is called the \emph{Greatest Lower Bound} of E or the \emph{infimum} of E, denoted
			\begin{displaymath}
			\beta=inf(E)
			\end{displaymath}
		\end{defn}

		\begin{defn}[least-upper-bound property] \label{def_least_upper_bound_property}
			An ordered set $S$ is said to have the \emph{least-upper-bound property} if the following is true:\\
			if $E \subset S$, $E$ is not empty, and $E$ is bounded above, then $sup(E)$ exists in $S$.
		\end{defn}

		\begin{defn}[greatest-lower-bound property] \label{def_greatest_lower_bound_property}
			An ordered set $S$ is said to have the \emph{greatest-lower-bound property} if the following is true:\\
			if $E \subset S$, $E$ is not empty, and $E$ is bounded below, then $inf(E)$ exists in $S$.
		\end{defn}

		\begin{thm} \label{thm_glb_lub_property}
			Suppose $S$ is an ordered set with the least-upper-bound property, $B \subset S$, $B$ is not empty, and $B$ is bounded below.\\
			Let $L$ be the set of all lower bounds of $B$. Then
			\begin{displaymath}
				\alpha=sup(L)
			\end{displaymath}
			exists in $S$, and $\alpha=inf(B)$.
		\end{thm}

		\begin{proof}
			Note that $\forall x \in L, y \in B, x \leq y$.\\
			$L$ is nonempty as $B$ is bounded below.\\
			$L$ is bounded above since $\forall x \in S \backslash L, \forall y \in L, x>y$.\\
			Since $S$ has the least-upper-bound property and $L \subset S$, $\exists \alpha=sup(L)$.\\
			The followings hold:
			\begin{itemize}
				\item $\alpha$ is a lower bound of $B$.
					\\($\because$) $\forall \gamma \in B, \gamma > \alpha$
				\item $\beta$ with $\beta > \alpha$ is not a lower bound of $B$
					\\($\because$)Since $\alpha$ is an upper bound of $L$, $\beta \notin L$.
			\end{itemize}
			Hence $\alpha=inf(B)$.
		\end{proof}
		
		\begin{coro} \label{coro_glb_lub_property_equiv}
			For all ordered sets, the Least Upper Bound property and the Greatest Lower Bound Porperty are equivalent.
		\end{coro}

		\subsection{Group}

			\begin{defn}[Group] \label{def_group}
				A \emph{group} is a set $G$ with a binary operation $\cdot$, denoted $(G,\cdot)$, which satisfies the following conditions:
				\begin{itemize}
					\item \textbf{Closure}: $\forall a,b \in G, a \cdot b \in G$
					\item \textbf{Associativity}: $\forall a,b,c \in G, (a \cdot b) \cdot c=a \cdot (b \cdot c)$
					\item \textbf{Identity}: $\exists e \in G, \forall a \in G, a \cdot e=e \cdot a=a$
					\item \textbf{Inverse}: $\forall a \in G, \exists a^{-1} \in G, a \cdot a^{-1}=a^{-1} \cdot a=e$
				\end{itemize}
			\end{defn}

		\begin{defn}[Semigroup] \label{def_semigroup}
			A \emph{semigroup} is $(G,\cdot)$, which satisfies Closure and Associativity.
		\end{defn}

		\begin{defn}[Monoid] \label{def_monoid}
			A \emph{monoid} is a semigroup $(G,\cdot)$ which also has identity.
		\end{defn}

		\begin{defn}[Abelian Group] \label{def_abelian_group}
			An \emph{Abelian Group} or \emph{Commutative Group} is a group $(G,\cdot)$ with the following property:
			\begin{itemize}
				\item \textbf{Commutativity}: $\forall a,b \in G, a \cdot b=b \cdot a$
			\end{itemize}
		\end{defn}

		\subsection{Ring}

			\begin{defn}[Ring] \label{def_ring}
				A \emph{Ring} is a set $R$ with two binary operations $+$ and $\cdot$, often called the addition and multiplication of the ring, denoted $(R,+,\cdot)$, which satisfies the following conditions:
				\begin{itemize}
					\item $(R,+)$ is an abelian group
					\item $(R,\cdot)$ is a semigroup
					\item \textbf{Distribution}: $\cdot$ is distributive with respect to $+$, that is, $\forall a,b,c \in R$:
					\begin{itemize}[label=-]
						\item $a \cdot (b + c)=(a \cdot b) + (a \cdot c)$
						\item $(a + b) \cdot c=(a \cdot c) + (b \cdot c)$
					\end{itemize}
				\end{itemize}
			The identity element of $+$ is often noted $0$.
			\end{defn}

		\begin{defn}[Ring with identity(1)] \label{def_ring_with_1}
			A \emph{Ring with identity} is a ring $(R,+,\cdot)$ of which $(R,\cdot)$ is a monoid. The identity element of $\cdot$ is often noted $1$.
		\end{defn}

		\begin{defn}[Commutative Ring] \label{def_commutative_ring}
			A \emph{commutative ring} is a ring $(R,+,\cdot)$ of which $\cdot$ is commutative.
		\end{defn}

		\begin{defn}[Zero Divisor] \label{def_zero_divisor}
			For a ring $(R,+,\cdot)$, let $0$ be the identity of $+$.\\
			$a,b\in R$, $a \neq 0$ and $b \neq 0$, if $a \cdot b=0$, $a,b$ are called the zero divisors of the ring.
		\end{defn}

		\begin{defn}[Integral Domain] \label{def_integral_domain}
			An \emph{integral domain} is a commutative ring $(R,+,\cdot)$ with 1 which does not have zero divisors.
		\end{defn}

		\subsection{Field}

		\begin{defn}[Field] \label{def_field}
			A \emph{Field} is a set $F$ with two binary operations $+$ and $\cdot$, often called the addition and multiplication of the field, denoted $(R,+,\cdot)$, which satisfies the following conditions:
			\begin{itemize}
				\item $(F,+,\cdot)$ is a ring
				\item $(F\backslash \{0\},\cdot)$ is a group
			\end{itemize}
			Alternatively, a Field may be defined with a set of \emph{Field Axioms} listed below:
			\begin{itemize}
					\item[(A)] \textbf{Axioms for Addition}
					\begin{itemize}
						\item[(A1)] \textbf{Closed under Addition}\\$\forall a,b \in F, a+b \in F$
						\item[(A2)] \textbf{Addition is Commutative}\\$\forall a,b \in F, a+b=b+a$
						\item[(A3)] \textbf{Addition is Associative}\\$\forall a,b,c \in F, (a+b)+c=a+(b+c)$
						\item[(A4)] \textbf{Identity of Addition}\\$\exists 0 \in F, \forall a \in F, 0+a=a$
						\item[(A5)] \textbf{Inverse of Addition}\\$\forall a \in F, \exists -a \in F, a+(-a)=0$
					\end{itemize}
					\item[(M)] \textbf{Axioms for Multiplication}
					\begin{itemize}
						\item[(M1)] \textbf{Closed under Multiplication}\\$\forall a,b \in F, a \cdot b \in F$
						\item[(M2)] \textbf{Multiplication is Commutative}\\$\forall a,b \in F, a \cdot b=b \cdot a$
						\item[(M3)] \textbf{Multiplication is Associative}\\$\forall a,b,c \in F, (a \cdot b) \cdot c=a \cdot (b \cdot c)$
						\item[(M4)] \textbf{Identity of Multiplication}\\$\exists 1 \in F, \forall a \in F, 1 \cdot a=a$
						\item[(M5)] \textbf{Inverse of Multiplication}\\$\forall a \in F\backslash\{0\}, \exists a^{-1} \in F, a \cdot a^{-1}=1$
					\end{itemize}
				\item[(D)] \textbf{Distributive Law}
				\\$\forall a,b,c \in F, (a+b) \cdot c=a \cdot c+b \cdot c$\\where $\cdot$ takes precedence over $+$.
			\end{itemize}
		\end{defn}
		
		\begin{thm}
			Let $F$ be a field. Let $0$ be the additive identity of $F$. Then, $\forall a \in F, 0 \cdot a = 0$
		\end{thm}
		
		\begin{defn}[Ordered Field] \label{def_ordered_field}
			An \emph{ordered field} is a field $F$ which is an ordered set, such that the order is compatible with the field operations, that is:
			\begin{itemize}
				\item $x+y<x+z$ if $x,y,z \in F$ and $y<z$
				\item $xy>0$ if $x,y \in F$, $x>0$ and $y>0$
			\end{itemize}
			\end{defn}
	
		\subsection{Polynomial Ring}
		\begin{defn}[Polynomial over a Ring] \label{def_polynomial}
			A polynomial $f(x)$ over the ring $(R,+,\cdot)$ is defined as
			\begin{displaymath}
				f(x)=\sum_{i=0}^{\infty}a_ix^i=a_0+a_1x^1+\cdots,a_i\in R
			\end{displaymath}
			where $a_i=0$ for all but finitely many values of $i$.\\
			The \emph{degree} of the polynomial $\deg(f)$ is defined as $\deg(f)=\max\{n|n\in\mathbb{N}, a_n\ne0 \}$.
			The \emph{leading coefficient} of the polynomial is defined as $a_{\deg(f)}$.
		\end{defn}

		\begin{defn}[Addition and Multiplication of Polynomials] \label{def_add_mult_polynomial}
			Let $f(x)=\sum_{i=0}^{\infty}a_ix^i$, $g(x)=\sum_{i=0}^{\infty}b_ix^i$, $a_i,b_i \in R$ be a polynomial over the ring $(R,+,\cdot)$. Define:
			\begin{gather*}
				f(x)+g(x)=\sum_{i=0}^{\infty}(a_i+b_i)x^i\\
				f(x)g(x)=\sum_{k=0}^{\infty}(c_k)x^k \text{ where } c_k=\sum_{i+j=k}a_ib_j
			\end{gather*}
		\end{defn}

		\begin{defn}[Polynomial Ring] \label{def_polynomial_ring}
			The set of polynomials over the ring $(R,+,\cdot)$, $R[x]=\{f(x)|f(x) \text{ is a polynomial over } R \}$ is called the \emph{Polynomial Ring(or Polynomials) over $R$}.
		\end{defn}

		\begin{thm}[Degree of Polynomial on Addition and Multiplication] \label{thm_add_mult_deg}
			Let $f(x),g(x) \in R[x]$ with $\deg(f)=n$, $\deg(g)=m$.
			\begin{itemize}
				\item $0 \le \deg(f+g) \le \max(\deg(f),\deg(g))$
				\item $\deg(fg) \le \deg(f)+\deg(g)$.
				\subitem If $(R,+,\cdot)$ is an integral domain, $\deg(fg) = \deg(f)+\deg(g)$
			\end{itemize}
		\end{thm}

		\begin{thm}[Relationship between a Ring and its Polynomial Ring] \label{thm_ring_polynomial_relationship}
			Let $(R,+,\cdot)$ be a ring and $R[x]$ the polynomials over $R$.
			\begin{enumerate}
				\item If $(R,+,\cdot)$ is a commutative ring with $1$, then $(R[x],+,\cdot)$ is a commutative ring with $1$.
				\item If $(R,+,\cdot)$ is a integral domain, then $(R[x],+,\cdot)$ is a integral domain.
			\end{enumerate}
		\end{thm}

		\begin{thm}[Division Algorithm for Polynomials over a Ring] \label{thm_polynomial_division_algorithm_ring}
			Let $(R,+,\cdot)$ be a commutative ring with $1$.\\
			Let $f(x),g(x) \in R[x]$, $g(x) \ne 0$ with the leading coefficient of $g(x)$ being invertible.\\
			Then, $\exists! q(x),r(x) \in R[x]$ such that
			\begin{displaymath}
				f(x)=q(x)g(x)+r(x)
			\end{displaymath}
			where either $r(x)=0$ or $\deg(r)<\deg(g)$.
		\end{thm}

		\begin{proof}
				Use induction on $\deg(f)$.\\
				1. $f(x)=0$ or $\deg(f)<\deg(g)$: $q(x)=0, r(x)=f(x)$\\
				2. $\deg(f)=\deg(g)=0$: $q(x)=f(x) \cdot g(x)^{-1}, r(x)=0$\\
				3. $\deg(f)\ge\deg(g)$:\\
				
				1) Existence\\
				Let $\deg(f)=n$, $\deg(g)=m$, $n>m$.\\
				Suppose the theorem holds for $\deg(f)<n$.\\
				Let $f(x)=a_0+a_1x^1+\cdots+a_nx^n$, $g(x)=b_0+b_1x^1+\cdots+b_mx^m$.\\
				Choose $f_1(x)=f(x)-(a_nb_m^{-1})x^{n-m}g(x)\in R[x]$.\\
				Since $\deg(f_1)<n$, $\exists q(x),r(x)\in R[x]$ so that $f_1(x)=g(x)q(x)+r(x)$, where $r(x)=0$ or $\deg(r)<\deg(g)$.\\
				$f_1(x)=f(x)-(a_nb_m^{-1})x^{n-m}g(x)=g(x)q(x)+r(x)$\\
				$f(x)=g(x)((a_nb_m^{-1})x^{n-m}+q(x))+r(x)$\\
				Hence such pair exists.\\
				
				2) Uniqueness\\
				Suppose $f(x)=g(x)q_1(x)+r_1(x)=g(x)q_2(x)+r_2(x)$.\\
				$g(x)(q_1(x)-q_2(x))=r_2(x)-r_1(x)$\\
				If $r_1 \ne r_2$, $\deg(g)>\deg(r_2-r_1)=\deg(g(q_1-q_2))$.\\
				Since $\deg(g(q_1-q_2))\ge\deg(g)$ if $q_1-q_2\ne0$, $q_1=q_2$, but if so, $r_1=r_2$.\\
				If $r_1=r_2$, trivially $q_1=q_2$.\\
				Hence they exist uniquely.
		\end{proof}
		
		\subsection{Vector Space}\label{chap_vector_space}
		\begin{defn}[Vector Space]
			A \emph{vector space} over a field(sometimes called the \emph{scalar} of the vector space) $F$ is a set $V$ together with two operations, addition($+:V\times V \rightarrow V$) and scalar multiplication($\cdot:F \times V \rightarrow V$), satisfying the following axioms:
			\begin{itemize}
				\item[(A)] \textbf{Axioms for Addition}
				\begin{itemize}
					\item[(A1)] \textbf{Closed under Addition}\\$\forall \bm{u} ,\bm{v} \in V, \bm{u}+\bm{v} \in V$
					\item[(A2)] \textbf{Addition is Commutative}\\$\forall \bm{u}, \bm{v} \in V, \bm{u}+\bm{v}=\bm{v}+\bm{u}$
					\item[(A3)] \textbf{Addition is Associative}\\$\forall \bm{u},\bm{v},\bm{w} \in v, (\bm{u}+\bm{v})+\bm{w}=\bm{u}+(\bm{v}+\bm{w})$
					\item[(A4)] \textbf{Identity of Addition(Zero vector)}\\$\exists \bm{0} \in V, \forall \bm{u} \in F, \bm{0}+\bm{u}=\bm{u}+\bm{0}=\bm{u}$
					\item[(A5)] \textbf{Inverse of Addition(Negative)}\\$\forall \bm{u} \in V, \exists -\bm{u} \in V, \bm{u}+(-\bm{u})=0$
				\end{itemize}
				\item[(M)] \textbf{Axioms for Scalar Multiplication}
				\begin{itemize}
					\item[(M1)] \textbf{Closed under Scalar Multiplication}\\$\forall k \in F, \bm{u} \in V, k \cdot \bm{u} \in V$
					\item[(M2)] \textbf{Scalar Multiplication is Distributive(1)}\\$\forall k \in F, \bm{u}, \bm{v} \in V, k \cdot (\bm{u}+\bm{v})=k \cdot \bm{u} + k \cdot \bm{v}$
					\item[(M3)] \textbf{Scalar Multiplication is Distributive(2)}\\$\forall k, m \in F, \bm{u} \in V, (k+m) \cdot \bm{u}=k \cdot \bm{u} + m \cdot \bm{u}$
					\item[(M4)] \textbf{Scalar Multiplication is Associative}\\$\forall k, m \in F, \bm{u} \in V, (km) \cdot \bm{u}=k \cdot (m \cdot \bm{u})$
					\item[(M5)] \textbf{Identity of Scalar Multiplication}\\$\exists 1 \in F, \forall \bm{u} \in V, 1 \cdot \bm{u}=\bm{u}$
				\end{itemize}
			\end{itemize}
			A vector space over $\mathbb{R}$ is called a \emph{real vector space}.
		\end{defn}
	
		\begin{thm}
			Let $V$ be a vector space over a field $F$. $\bm{u} \in V$, $k \in F$, $0$ the additive identity of $F$, $1$ the multiplicative identity of $F$, $\bm{0}$ the additive identity of $V$. Then, the followings hold:
			\begin{itemize}
				\item $0 \cdot \bm{u} = \bm{0}$
				\item $k \cdot \bm{0} = \bm{0}$
				\item $-1 \cdot \bm{u} = -\bm{u}$
				\item If $k \cdot \bm{u} = \bm{0}$, then $k=0$ or $\bm{u}=\bm{0}$.
			\end{itemize}
		\end{thm}
		
		\begin{defn}[Subspace of a Vector Space]
			A subset $W$ of a vector space $V$ is called a \emph{subspace} of $V$ if $W$ is a vector space under the addition and scalar multiplication defined on $V$.
		\end{defn}

		\begin{thm}
			If $W$ is a set of one or more vectors in a vector space $V$ over the field $F$, then $W$ is a subspace of $V$ iff the following conditions hold:
			\begin{itemize}
				\item $\forall \bm{u}, \bm{v} \in W, \bm{u}+\bm{v} \in W$
				\item $\forall k \in F, \bm{u} \in W, k \cdot \bm{u} \in W$
			\end{itemize}
		\end{thm}
	
		\begin{thm}
			If $W_1, W_2, \dots , W_r$ are subspaces of a vector space $V$, then $\cap_{i=1}^{r}W_i$ is also a subspace of $V$.
		\end{thm}
	
		\begin{defn}[Linear Combination] \label{def_linear_combination_vector}
			If $\bm{w}$ is a vector in a vector space $V$ over the field $F$, then $\bm{w}$ is said to be a \emph{Linear Combination} of the vectors $\bm{v_1}, \bm{v_2}, \dots, \bm{v_r} \in V$ if $\bm{w}$ can be expressed in the form $\bm{w}=\sum_{i=1}^{r}k_i\bm{v_i}$, where $k_1, k_2, \dots, k_r \in F$. These scalars are called the \emph{coefficients} of the linear combination.
		\end{defn}
	
		\begin{defn}[Span]
			The subspace of a vector space $V$ that is formed from all possible linear combinations of the vectors in a nonempty set $S$ is called the \emph{Span} of $S$, and we say that the vectors in $S$ \emph{span} that subspace.
		\end{defn}
	
		\begin{thm}
			If $S=\{\bm{v_1}, \bm{v_2}, \dots, \bm{v_r}\}$ and $S'=\{\bm{w_1}, \bm{w_2}, \dots, \bm{w_k}\}$ are nonempty sets of vectors in a vector space $V$, then $\verb|span|(S)=\verb|span|(S')$ iff each vector in $S$ is a linear combination of those in $S'$ and vice versa.
		\end{thm}

		\begin{defn}[Basis]\label{def_basis}
			If $V$ is any vector space and $S=\{\bm{v_1}, \bm{v_2}, \dots, \bm{v_r}\}$ is a finite set of linearly independent vectors in $V$ which spans $V$, then $S$ is called a \emph{basis} for $V$.
		\end{defn}
	
		\begin{thm}
			All bases for a finite-dimensional vector space have the same number of vectors.
		\end{thm}
		
		\begin{defn}[Dimension]
			The \emph{dimension} of a finite-dimensional vector space $V$, denoted by $\dim (V)$, is defined to be the number of vectors in a basis for $V$. In addition, the zero vector space is defined to have dimension zero.
		\end{defn}
		
		\begin{thm}[Plus/Minus Theorem]
			Let $S$ be a nonempty set of vectors in a vector space $V$.
			\begin{itemize}
				\item If $S$ is a linearly independent set, and if $\bm{v}$ is a vector in $V$ that is outside of $span(S)$, then the set $S \cup \{\bm{v}\}$ that results by inserting $\bm{v}$ into $S$ is still linearly independent.
				\item If $\bm{v} \in S$ is expressible as a linear combination of the vectors in $S-\{\bm{v}\}$, then $span(S)=span(S-\{\bm{v}\})$.
			\end{itemize}
		\end{thm}
		
		\begin{thm}
			Let $V$ be an $n$ dimensional vector space, and let $S$ be a set in $V$ with exactly $n$ vectors. Then $S$ is a basis for $V$ iff $span(S)=V$ or $S$ is linearly independent.
		\end{thm}
		
		\begin{thm}
			Let $S$ be a finite set of vectors in a finite dimensional vector space $V$.
			\begin{itemize}
				\item If $S$ spans $V$ but is not a basis for $V$, then $S$ can be reduced to a basis for $V$ by removing appropriate vectors from $S$.
				\item If $S$ is a linearly independent set that is not already a basis for $V$, then $S$ can be enlarged to a basis for $V$ by inserting appropriate vectors into $S$.
			\end{itemize}
		\end{thm}
		
		\begin{thm}
			If $W$ is a subspace of a finite-dimensional vector space $V$, then:
			\begin{itemize}
				\item $W$ is finite dimensional
				\item $dim(W) \le dim(V)$
				\item $W=V$ iff $dim(W)=dim(V)$.				\item $A$ is positive definite iff all eivenvalues of $A$ are positive.
			\end{itemize}
		\end{thm}
	
		\begin{thm}[Uniqueness of Basis Representation]\label{thm_coordinate_unique}
			If $S=\{\bm{v_1}, \bm{v_2}, \dots, \bm{v_r}\}$ is a basis for a vector space $V$, then every vector $\bm{v}$ in $V$ can be expressed in the form $\bm{v}=c_1\bm{v_1}+c_2\bm{v_2}+\dots+c_r\bm{v_r}$ in exactly one way.
		\end{thm}
	
		\begin{defn}[Coordinate]\label{def_coordinate}
			Let $S=\{\bm{v_1}, \bm{v_2}, \dots, \bm{v_r}\}$ be a basis for a vector space $V$ over the field $F$, and $\bm{v}=c_1\bm{v_1}+c_2\bm{v_2}+\dots+c_r\bm{v_r}$ is the expression for a vector $\bm{V}$ in terms of the basis $S$, then the scalars $c_1,c_2,\dots,c_n$ are called the \emph{coordinates} of $\bm{v}$ relative to the basis $S$. The vector $(c_1,c_2,\dots,c_n)$ in $F^n$ constructed from these coordinates is called the \emph{coordinate vector of $\bm{v}$ relative to $S$}, denoted by $(\bm{v})_S=(c_1,c_2,\dots,c_n)$.
		\end{defn}
		
		\subsubsection{Linear Transformation}
		%TODO
		
		\subsection{Inner Product Space}\label{chap_inner_product_space}
			\begin{defn}[Inner Product Space]
				An \emph{inner product} on a real vector space $V$ is a function that associates a real number $\left<\bm{u},\bm{v}\right>$ with each pair of vectors in $V$ in a such way that the following axioms are satisfied for all vectors $\bm{u},\bm{v},\bm{w}\in V$ and all scalars $k$.
				\begin{enumerate}
					\item $\left<\bm{u},\bm{v}\right>=\left<\bm{v},\bm{u}\right>$ [Symmetry Axiom]
					\item $\left<\bm{u}+\bm{v},\bm{w}\right>=\left<\bm{u},\bm{w}\right>+\left<\bm{v},\bm{w}\right>$ [Additivity Axiom]
					\item $\left<k\bm{u},\bm{v}\right>=k\left<\bm{u},\bm{v}\right>$ [Homogeneity Axiom]
					\item $\left<\bm{v},\bm{v}\right> \ge 0$ and $\left<\bm{v},\bm{v}\right>=0$ iff $\bm{v}=\bm{0}$. [Positivity Axiom]
				\end{enumerate}
				A real vector space with an inner product is called a \emph{real inner product space}.
			\end{defn}
			
			\begin{defn}[Norm and Distance]
				If $V$ is a real inner product space, then the \emph{norm} or \emph{length} of a vector $\bm{v}$ in $V$, denoted by $\|\bm{v}\|$, is defined by
				\begin{displaymath}
					\|\bm{v}\|=\sqrt{\left<\bm{v},\bm{v}\right>}
				\end{displaymath}
				and the \emph{distance} between two vectors, denoted by $d(\bm{u},\bm{v})$, is defined by
				\begin{displaymath}
					d(\bm{u},\bm{v})=\|\bm{u}-\bm{v}\|=\sqrt{\left<\bm{u}-\bm{v},\bm{u}-\bm{v}\right>}
				\end{displaymath}
				A vector of norm 1 is called a \emph{unit vector}.
			\end{defn}
			
			If $V$ is an inner product space, then the set of points in $V$ that satisfy $\|\bm{u}\|=1$ is called the \emph{unit sphere} or sometimes the \emph{unit circle} in $V$.
			
			\begin{thm}
				If $\bm{u}$ and $\bm{v}$ are vectors in a real inner place $V$ and if $k$ is a scalar, then:
				\begin{itemize}
					\item $\|\bm{v}\| \ge 0$ with equality iff $\bm{v}=\bm{0}$
					\item $\|k\bm{v}\|=|k|\|\bm{v}\|$
					\item $d(\bm{u},\bm{v})=d(\bm{v},\bm{u})$
					\item $d(\bm{u},\bm{v}) \ge 0$ with equality iff $\bm{u}=\bm{v}$
				\end{itemize}
			\end{thm}
			
			\begin{defn}
				If $\bm{u}, \bm{v}, \bm{w} \in V$ and if $k$ is a scalar, then:
				\begin{itemize}
					\item $\left< \bm{0} , \bm{v} \right>=\left< \bm{v} , \bm{0} \right>=0$
					\item $\left< \bm{u} , \bm{v}+\bm{w} \right>=\left< \bm{u} , \bm{v} \right>+\left< \bm{u} , \bm{w} \right>$
					\item $\left< \bm{u} , \bm{v}-\bm{w} \right>=\left< \bm{u} , \bm{v} \right>-\left< \bm{u} , \bm{w} \right>$
					\item $\left< \bm{u}-\bm{v} , \bm{w} \right>=\left< \bm{u} , \bm{w} \right>-\left< \bm{v} , \bm{w} \right>$
					\item $k\left< \bm{u} , \bm{v} \right>=\left< \bm{u} , k\bm{v} \right>$
				\end{itemize}
			\end{defn}
		
\end{document}
	
		\chapter{Number Theory}
		    \input{./chapters/mathematics/number_theory.tex}
	
		\chapter{Analysis}
		    %\documentclass{report}

\begin{document}
    \section{Metric Spaces}
    
    \subsection{Topology of Metric Spaces}
    
    \begin{defn}[Metric Space] \label{def_metric_space}
        A set $X$ equipped with a function $d:X \times X \rightarrow \mathbb{R}$ is a \emph{metric space} if $d$ satisfies, for all $p,q,r \in X$:\begin{enumerate}
            \item $d(p,q)>0$ for $p \neq q$, and $d(p,p)=0$.
            \item $d(p,q)=d(q,p)$.
            \item $d(p,q) \leq d(p,r)+d(r,q)$. This inequality is called the \emph{triangle inequality}.
        \end{enumerate}
        The elements of $X$ are called \emph{point}s. The function $d$ is called a \emph{metric}.
    \end{defn}
    
    \begin{defn} \label{def_analysis_top}
        Let $X$ be a metric space, $E \subseteq X$, and $p \in X$. \begin{itemize}
            \item A \emph{neighborhood} of $p$, denoted $N_r(p)$, is $\{q \in X | d(p,q) \leq r \}$, where $r>0$.
            \item $p$ is a \emph{limit point} of $E$ if every neighborhood of $p$ contains $q \in E$ different from $p$. The set of all limit points of $E$ is denoted $E'$.
            \item The \emph{boundary} of $E$ is (TODO)
            \item $p$ is an \emph{interior point} of $E$ if there is a neighborhood of $p$ that is contained in $E$.
            \item $p$ is an \emph{isolated point} of $E$ if $p \in E$ and $p$ is not a limit point of $E$.
            \item $E$ is \emph{open} if every point in $E$ is an interior point.
            \item $E$ is \emph{closed} if if every limit point of $E$ is in $E$.
            \item $E$ is \emph{bounded} if there is a neighborhood of some $p$ that contains $E$.
            \item $E$ is \emph{dense} if every point of $X$ is a limit point of $E$ or a point of $E$.
        \end{itemize}
    \end{defn}
    
    Here is a figure demonstrating these notions in the space $\mathbb{R}$ with the metric $d(x,y) = |x-y|$ and $E = [0,1) \cap \{2\}$:
    
    (TODO)
    
    Note that a set can be both open and closed. For example, an empty set is (vacuously) both open and closed. $X$ itself is also both open and closed. The notions in topology will be covered in greater detail in the Topology chapter.
    
    From now on, assume $X$ is always a metric space with the metric $d$, and $E \subseteq X$, unless stated otherwise.
    
    \begin{prop} \label{prop_analysis_top} \begin{enumerate} \item[]
        \item A neighborhood is open.
        \item If $p$ is a limit point of $E$, then every neighborhood contains infinitely many points of $E$.
        \item $E$ is open iff $E^C$ is closed.
        \item $E$ is closed iff $E^C$ is open.
    \end{enumerate} \end{prop}
    
    \begin{proof}
    1. Let $q \in N_r(p)$. Then $N_{r-d(p,q)}(q) \subseteq N_r(p)$ because, if $x \in N_{r-d(p,q)}(q)$, then $d(p,x) \leq d(p,q)+d(q,r) < d(p,q)+r-d(p,q) = r$ so $x \in N_r(p)$.
    
    2. Suppose some neighborhood $N_r(p)$ contains only finitely many points of $E$, namely $x_1$, $\cdots$, $x_k$. Let $r=min_{i=1}^k d(p,x_i)$. Then $N_r(p)$ contains no points of $E$, contradiction.
    
    3. TODO
    
    4. $E = (E^C)^C$.
    \end{proof}
    
    \subsection{Compact Sets}
    
    \section{Sequences}
    
    \section{Series}
    
    \section{Continuity}
    
    \section{Differentiation}
    
    \section{Integral}
    
    \section{Sequences and Series of Functions}

\end{document}
		
		\chapter{Linear Algebra}
		    \documentclass{report}

\begin{document}
The target of Linear Algebra is to solve a system of homogenous linear equations. To do so, we deal with vectors and matrices.
	\section{Vector Spaces}
		For the definitions on vector spaces, subspaces, and bases, refer to the chapter \ref{chap_vector_space}.
		\subsection{Linear Independence}
		We now define linear independence, one of the most important concepts utilized in linear algebra.
		\begin{defn}[Linear Independence]
			if $S=\{\vec{v_1}, \vec{v_2}, \dots, \vec{v_r}\}$ is a nonempty set of vectors in a vector space $V$, then the vector equation $k_1\vec{v_1}+k_2\vec{v_2}+\dots+k_r\vec{v_r}=\vec{0}$ has at least one solution, namely, $k_1=0, k_2=0, \dots, k_r=0$, the \emph{trivial solution}. If this is the only solution, then $S$ is said to be a \emph{linearly independent set}. If there are solutions in addition to the trivial solution, then $S$ is said to be \emph{linearly dependent}.
		\end{defn}
		
		\begin{thm}
			Let $S=\{\vec{v_1}, \vec{v_2}, \dots, \vec{v_r}\}$ be a set of vectors in $\mathbb{R}^n$. If $r>n$, then $S$ is linearly dependent.
		\end{thm}
		
		\subsection{Orthogonality}
		\begin{defn}[Euclidean Inner Product]
			Let $\vec{u}=(u_1,u_2,\dots,u_n)$ and $\vec{v}=(v_1,v_2,\dots,v_n)$ in $\mathbb{R}^n$. The inner product of the two vectors $\vec{u}$ and $\vec{v}$ is defined as
			\begin{displaymath}
				\vec{u}\cdot\vec{v}=\sum_{i=1}^{n}u_iv_i=u_1v_1+u_2v_2+\dots+u_nv_n
			\end{displaymath}
		\end{defn}
		
		\begin{defn}[Norm]
			The \emph{norm} of $\vec{u}=(u_1,u_2,\dots,u_n)$ in $\mathbb{R}^n$, denoted $\|\vec{u}\|$, is defined by
			\begin{displaymath}
			\|\vec{u}\|=\sqrt{\vec{u} \cdot \vec{u}}=\sqrt{\sum_{i=1}^{n}u_i^2}=\sqrt{u_1^2+u_2^2+\dots+u_n^2}
			\end{displaymath}
		\end{defn}
		
		\begin{defn}[Distance]
			If $\vec{u}=(u_1,u_2,\dots,u_n)$ and $\vec{v}=(v_1,v_2,\dots,v_n)$ are vectors in $\mathbb{R}^n$, then the \emph{distance} between $\vec{u}$ and $\vec{v}$, denoted $d(\vec{u},\vec{v})$, and define it to be:
			\begin{displaymath}
				d(\vec{u},\vec{v})=\|\vec{u}-\vec{v}\|=\sqrt{(u_1-v_1)^2+(u_2-v_2)^2+\cdots+(u_n-v_n)^2}
			\end{displaymath}
		\end{defn}
		
		\begin{defn}[Unit vectors]
			A vector $\vec{u}$ in $\mathbb{R}^n$ is said to be a unit vector iff $\|\vec{u}\|=1$.
		\end{defn}
		
		\begin{defn}[Angle]
			The \emph{angle} between two nonzero vectors $\vec{u}$ and $\vec{v}$ in $\mathbb{R}^n$ is defined by
			\begin{displaymath}
				\theta=\cos^{-1}(\frac{\vec{u}\cdot\vec{v}}{\|\vec{u}\|\|\vec{v}\|})
			\end{displaymath}
		\end{defn}
		
		\begin{thm}[Cauchy-Schwarz Inequality]
			If $\vec{u}=(u_1,u_2,\dots,u_n)$ and $\vec{v}=(v_1,v_2,\dots,v_n)$ are vectors in $\mathbb{R}^n$, then $\left|\vec{u}\cdot\vec{v}\right| \le \|u\|\|v\|$. In terms of components:
			\begin{displaymath}
				\left|u_1v_1+u_2v_2+\cdots+u_nv_n\right| \le \left(u_1^2+u_2^2+\cdots+u_n^2\right)^{1/2} \left(v_1^2+v_2^2+\cdots+v_n^2\right)^{1/2}
			\end{displaymath}
		\end{thm}
		
		\begin{thm}[Triangle Inequality]
			If $\vec{u}, \vec{v}, \vec{w}$ are vectors in $\mathbb{R}^n$, then:
			\begin{itemize}
				\item $\|\vec{u}+\vec{v}\| \ge \|\vec{u}\|+\|\vec{v}\|$: Triangle Inequality for Vectors
				\item $d(\vec{u},\vec{v}) \ge d(\vec{u},\vec{v})+d(\vec{w},\vec{v})$: Triangle Inequality for Distances
			\end{itemize}
		\end{thm}
		
		\begin{thm}[Equations for Vectors within the Euclidean Space]
			If $\vec{u}$ and $\vec{v}$ are vectors in $\mathbb{R}^n$
			\begin{itemize}
				\item $\|\vec{u}+\vec{v}\|^2+\|\vec{u}-\vec{v}\|^2=2\left(\|u\|^2+\|v\|^2\right)$: Parallelogram Equation for Vectors
				\item $\vec{u}\cdot\vec{v}=\frac{1}{4}\|\vec{u}+\vec{v}\|^2-\frac{1}{4}\|\vec{u}-\vec{v}\|^2$
			\end{itemize}
		\end{thm}
		
		\begin{defn}[Orthogonal Vectors]
			Two nonzero vectors $\vec{u}$ and $\vec{v}$ in $\mathbb{R}^n$ are said to be \emph{orthogonal} or \emph{perpendicular} if $\vec{u} \cdot \vec{v} = 0$.
		\end{defn}
		
		\begin{defn}[Orthogonal set]
			A nonempty set of vectors in $\mathbb{R}^n$ is called an \emph{orthogonal set} if all pairs of distinct vectors in the set are orthogonal. If they are also all unit vectors, it is called an \emph{orthonormal set}.\\
			In other words, for a set $\{u_1, u_2, \dots, u_n\}$ to be orthogonal:
			\begin{displaymath}
				u_i \cdot u_j
				\begin{cases}
					\|u_i\|^2 & i=j\\
					0 & i \ne j
				\end{cases}
			\end{displaymath}
			And for the set to be orthonormal, in addition to above, $\forall i \in \{1, 2, \dots n\}, \|u_i\|=1$.
		\end{defn}
		
		\begin{thm}[Projection Theorem]
			If $\vec{u}$ and $\vec{a}$ are vectors in $\mathbb{R}^n$ and if $\vec{a} \ne \vec{0}$, then $\vec{u}$ can be expresses in exactly one way in the form $\vec{u}=\vec{w_1}+\vec{w_2}$, where $\exists k \in \mathbb{R}, \vec{w_1}=k\vec{a}$ and $\vec{a} \cdot \vec{w_2}=0$.
		\end{thm}
		
		In the theorem above, the vector $\vec{w_1}$ is called the \emph{orthogonal projection of $\vec{u}$ on $\vec{a}$} or \emph{vector component of $\vec{u}$ along $\vec{a}$}, and the vector $\vec{w_2}$ is called the \emph{vector component of $\vec{u}$ orthogonal to $\vec{a}$}.
		
		\begin{defn}[Projection]
			Let $\vec{u}$ and $\vec{a} \ne \vec{0}$ be vectors in $\mathbb{R}^n$. The \emph{orthogonal projection of $\vec{u}$ on $\vec{a}$}, denoted $\verb|proj|_{\vec{a}}\vec{u}$
		\end{defn}
	
	\section{Matrix}
		\subsection{Matrices and its operations}
			\begin{defn}[Matrix]
				A \emph{matrix} is a rectangular array of numbers. The numbers in the array are called the \emph{entries} in the matrix.
			\end{defn}
		
			Equality, addition, and subtraction can only be defined on same-sized matrices, and is defined elementwise; scalar multiplication is also defined elementwise.
			
			\begin{defn}[Matrix Multiplication]
				If $A$ is an $m \times r$ matrix and $B$ is an $r \times n$ matrix, then the \emph{product} $AB$ is the $m \times n$ matrix whose entries are determined as follows:
				The entry of $AB$ on row $i$ and column $j$, multiply the corresponding entries from the row $i$ from $A$ and column $j$ from $B$, then add them all together.
			\end{defn}
		
			Matrices of the same size may be used in a linear combination, just like vectors[\ref{def_linear_combination_vector}].
			
			\begin{defn}[Linear Combination of a Matrix]
				If $A_1, A_2, \dots, A_r$ are matrices of the same size, and if $c_1, c_2, \dots, c_r$ are scalars, then an expression of the form
				\begin{displaymath}
					c_1A_1+c_2A_2+\cdots+c_rA_r
				\end{displaymath}
				is called a \emph{linear combination} of $A_1, A_2, \dots, A_r$ with coefficients $c_1, c_2, \dots, c_r$.
			\end{defn}
			
			\begin{thm}
				If $A$ is an $m \times n$ matrix and if $\vec{x}$ is an $n \times 1$ column vector, then the product $A\vec{x}$ can be expressed as a linear combination of the column vectors of $A$ in which the coefficients are the entries of $\vec{x}$.
			\end{thm}
			
			\begin{defn}[Transpose]
				For any $m \times n$ matrix, then the \emph{transpose} of $A$, denoted by $A^T$, is defined to be the $n \times m$ matrix that results by interchanging the rows and columns of $A$; that is, the first column of $A^T$ is the first row of $A$ and so forth.
			\end{defn}
			
			\begin{defn}[Trace]
				For a square matrix $A$, the \emph{trace} of $A$, denoted $tr(A)$, is defined to be the sum of the entries on the main diagonal of $A$.
			
		\end{defn}
	
	\section{Inverse}
		\subsection{Elementary Row Operations and Matrices}
		
		\begin{defn}[Elementary Row Operations]\label{def_elementary_row_operations}
			The following three operations are said to be the \emph{elementary row operations} on a matrix:
			\begin{enumerate}
				\item Multiply a row through by a nonzero constant.
				\item Interchange two rows.
				\item Add a constant times one row to another.
			\end{enumerate}
		\end{defn}
		
		\begin{defn}[Elementary Row Matrices]
			An $n \times n$ matrix is called an \emph{elementary matrix} if it can be obtained from the $n \times n$ identity matrix $I_n$ by performing a single elementary row operation.
		\end{defn}
		
		\begin{thm}[Elementary Row Operations and Elementary Row Matrices]
			If the elementary matrix $E$ results from performing a certain row operation on $I_m$ and $A$ is an $m \times n$ matrix, then the product $EA$ is the matrix that results when this same row operation is performed on $A$.
		\end{thm}
		
		\begin{defn}[Reduced-row Echelon Form]
			A matrix that is in its \emph{reduced-row echelon form(rref)} has the following properties:
			\begin{enumerate}
				\item If a row does not consist entirely of zeroes, then the first nonzero number in the row is a 1. We call this a \emph{leading 1}.
				\item If there are any rows that consist entirely of zeroes, then they are grouped together at the bottom of the matrix.
				\item In any two successive rows that do not consist entirely of zeroes, the leading 1 in the lower row occurs farther to the right than the leading 1 in the higher row.
				\item Each column that contains a leading 1 has zeroes everywhere else in that column
			\end{enumerate}
			A matrix that has the first three properties is said to be in \emph{row echelon form}.
		\end{defn}
		
		\begin{thm}
			If $R$ is the reduced row echelon form of an $n \times n$ matrix $A$, then either $R$ has a row of zeroes or $R$ is the identity matrix $I_n$.
		\end{thm}
		
		There are two important facts on echelon forms:
		\begin{enumerate}
			\item Every matrix has a unique rref.
			\item Row echelon forms are not unique, but, they have the same:
			\begin{itemize}
				\item number of zero rows
				\item positions of leading 1's
					\subitem the positions are called the \emph{pivot positions} of A
					\subitem the columns are called the \emph{pivot column} of A
			\end{itemize}
		\end{enumerate}
		
		\begin{mthd}[Gauss-Jordan Elimination]\label{mthd_gauss_jordan_elim}
			This method will use elementary row operations and through two phases, forward and backward phases, reduces a matrix into its reduced row echelon form.
			\begin{enumerate}[label=Phase \arabic*.]
				\item Forward Phase\footnote{If only this phase is used to produce a row echelon form, this is called the Gaussian elimination.}
				\begin{enumerate}[label=Step \arabic*.]
					\item Locate the leftmost column that does not consist entirely of zeroes.
					\item Interchange the top row with another row, if necessary, to bring a nonzero entry to the top of the column found in Step 1.
					\item Multiply the first row by a constant so that it has a leading 1.
					\item Add suitable multiples of the top row to the rows below so that all entries below the leading 1 become zeroes.
					\item Reapply Step 1, ignoring the upper rows until the entire matrix is in row echelon form.
				\end{enumerate}
				\item Backward Phase
				\begin{enumerate}[label=Step \arabic*.]
					\setcounter{enumii}{6}
					\item Beginning with the last nonzero row and working upward, add suitable multiples of each row to the rows above to make the entries above the leading 1's to 0.
				\end{enumerate}
			\end{enumerate}
		\end{mthd}
		
		\subsection{Finding the Inverse for a Matrix}
		\begin{defn}[Inverse]
			If $A$ is a square matrix, and if a matrix $B$ of the same size can be found so that $AB=BA=I$, then $A$ is said to be \emph{invertible} or \emph{nonsingular} and $B$ is called an \emph{inverse} of $A$, denoted by $A^{-1}$. If no such matrix $B$ can be found, then $A$ is said to be \emph{singular} or \emph{non-invertible}.
		\end{defn}
		
		\begin{thm}
			If $B$ and $C$ are both inverses of the matrix $A$, then $B=C$.
		\end{thm}
		
		\begin{thm}[Inverse of a 2-by-2 matrix]\label{thm_inv_2_by_2}
			The matrix
			\begin{displaymath}
				A=
				\begin{bmatrix}
					a & b \\ c & d
				\end{bmatrix}
			\end{displaymath}
			is invertible iff $ad-bc\ne0$, in which case the inverse is given by:
			\begin{displaymath}
				A^{-1}=\frac{1}{ad-bc}
				\begin{bmatrix}
					d & -b \\ -c & a
				\end{bmatrix}
			\end{displaymath}
		\end{thm}
		
		\begin{thm}
			If $A$ and $B$ are invertible matrices with the same size, then $AB$ is invertible and $(AB)^{-1}=B^{-1}A^{-1}$.\\
			In general, a product of any number of invertible matrices is invertible, and the inverse of the product is the product of the inverses in the reverse order.
		\end{thm}
		
		\begin{thm}
			If $A$ is invertible, then $A^T$ is also invertible, and $(A^T)^{-1}=(A^{-1})^T$.
		\end{thm}
		
		\begin{thm}
			Every elementary matrix is invertible, and the inverse is also an elementary matrix.
		\end{thm}
		
		\begin{mthd}[Inversion Algorithm]
			To find the inverse of an invertible matrix $A$, find a sequence of elementary row operations that reduces $A$ to the identity and then perform that same sequence of operations on $I_n$ to obtain $A^{-1}$.
			
			For easier approach, simply use Gauss-Jordan Elimination[\ref{mthd_gauss_jordan_elim}] to the augmented matrix $\left[A|I_n\right]$ so that it becomes $\left[I_n|A^{-1}\right]$.
		\end{mthd}
	
	\section{Determinants}
		Recall from [\ref{thm_inv_2_by_2}] that the $2 \times 2$ matrix $A=\begin{bmatrix} a & b \\ c & d \end{bmatrix}$ is invertible iff $ad-bc \ne 0$. The term $ad-bc$ is the determinant of the matrix $A$. \emph{Determinant} is a scalar value that can be computed from the elements of a square matrix which encodes certain properties of the matrix.
		\subsection{Calculating Determinants}
			There are two methods to calculate the determinant.
			\subsubsection{Method of Cofactor Expansion}
				\begin{defn}[Minors and Cofactors]
					Let $A$ be a square matrix. Then the \emph{minor of entry $a_{ij}$}, denoted by $M_ij$, is defined to be the determinant of the submatrix that remains after the i-th row and j-th column are deleted from $A$. The number $C_{ij}=(-1)^{i+j}M_{ij}$ is called the \emph{cofactor of entry $a_{ij}$}.
				\end{defn}
				
				\begin{defn}[Adjoint]
					If $A$ is $n \times n$ matrix and $C_{ij}$ is the cofactor of $a_{ij}$, then the matrix
					\begin{displaymath}
					\begin{bmatrix}
						C_{11} & C_{12} & \cdots & C_{1n} \\
						C_{21} & C_{22} & \cdots & C_{2n} \\
						\vdots & \vdots &        & \vdots \\
						C_{n1} & C_{n2} & \cdots & C_{nn} \\
					\end{bmatrix}
					\end{displaymath}
					is called the \emph{matrix of cofactors from $A$}. The transpose of this matrix is called the \emph{adjoint of $A$}, denoted by adj$(A)$.
				\end{defn}
				
				\begin{defn}[Determinant]
					If $A$ is an $n \times n$ matrix, then the number obtained by multiplying the entries in any row or column of $A$ by the corresponding cofactors and adding the resulting products is called the \emph{determinant} of $A$, and the sums themselves are called \emph{cofactor expansions} of $A$.
					
					The cofactor expansion along the j-th column is as follows:
					\begin{displaymath}
						\det(A)=a_{1j}C_{1j}+a_{2j}C_{2j}+\cdots+a_{nj}C_{nj}
					\end{displaymath}
					
					and the cofactor expansion along the i-th row is as follows:
					\begin{displaymath}
						\det(A)=a_{i1}C_{i1}+a_{i2}C_{i2}+\cdots+a_{in}C_{in}
					\end{displaymath}
				\end{defn}
				
				\begin{thm}\label{thm_triangular_determinant}
					If $A$ is an $n \times n$ triangular matrix, then $\det(A)$ is the product of entries on the main diagonal of the matrix; that is, $\det(A)=a_{11}a_{22}\cdots a_{nn}$.
				\end{thm}
			
			\subsubsection{Method of Elementary Row Operations}
				This section presents a series of theorems that can be proven with the cofactor expansion formula that will suffice by themselves, paired with the theorem for determinants for triangular matrices[\ref{thm_triangular_determinant}](or simply the fact that $\forall n, \det(I_n)=1$), to find the determinant by continuously applying elementary row operations to the target matrix.
				\begin{thm}
					Let $A$ be a square matrix. If $A$ has a row of zeroes or a column of zeroes, then $\det(A)=0$.
				\end{thm}
				
				\begin{thm}
					Let $A$ be a square matrix. Then $\det(A)=\det(A^T)$.
				\end{thm}
				
				\begin{thm}\label{thm_row_ops_det}
					Let $A$ be an $n \times n$ matrix.
					\begin{enumerate}
						\item If $B$ is the matrix that results when a single row or single column or $A$ is multiplies by a scalar $k$, then $\det(B)=k\det(A)$.
						\item If $B$ is the matrix that results when two rows or two columns of $A$ are interchanged, then $\det(B)=-\det(A)$.
						\item If $B$ is the matrix that results when a multiple of one row of $A$ is added to another row or when a multiple of one column is added another column, then $\det(B)=\det(A)$.
					\end{enumerate}
				\end{thm}
				
				\begin{coro}
					Let $E$ be an $n \times n$ matrix.
					\begin{enumerate}
						\item If $E$ results from multiplying a row of $I_n$ by a nonzero number $k$, then $\det(E)=k$.
						\item If $E$ results from interchanging two rows of $I_n$, then $det(E)=-1$.
						\item If $E$ results from adding a multiple of one row of $I_n$ to another, then $\det(E)=1$.
					\end{enumerate}
				\end{coro}
				
				\begin{thm}
					If $A$ is a square matrix with two proportional rows or two proportional columns, then $\det(A)=0$.
				\end{thm}
				
				Often times, the method of elementary row operations may be applied partially to assist with cofactor expansion formula.
			
		\subsection{Properties of Determinants}
			\begin{thm}
				Let $A$, $B$, and $C$ be $n \times n$ matrices that differ only in a single row, say the r-th row, and assume that the r-th row of $C$ can be obtained by adding corresponding entries in the r-th row of $A$ and $B$. Then, $\det(C)=\det(A)+\det(B)$.
				
				The same result holds for columns.
			\end{thm}
			
			Pairing the theorem above with theorem \ref{thm_row_ops_det}'s first fact, we can say that the determinant is a linear function of each row separately.
			
			\begin{thm}
				A square matrix $A$ is invertible iff $\det(A) \ne 0$.
			\end{thm}
			
			\begin{thm}
				If $A$ and $B$ are square matrices of the same size, then $\det(AB)=\det(A)\det(B)$.
			\end{thm}
			
			\begin{thm}
				If $A$ is invertible, then
				\begin{displaymath}
					\det(A^{-1})=\frac{1}{\det(A)}
				\end{displaymath}
			\end{thm}
			
			\begin{thm}[Inverse of a Matrix using its Adjoint]
				If $A$ is an invertible matrix, then
				\begin{displaymath}
					A^{-1}=\frac{1}{\det(A)}\verb|adj|(A)
				\end{displaymath}
			\end{thm}
			
			\begin{thm}[Cramer's Rule]
				If $A\bf{x}=\bf{b}$ is a system of n linear equations such that $det(A)\ne0$, then the system has a unique solution. The solution is:
				\begin{displaymath}
					\forall i, x_i=\frac{\det(A_i)}{\det(A)}
				\end{displaymath}
				where $A_i$ is the matrix obtained by replacing the entries in the j-th column of $A$ by the entries in the matrix $\bf{b}$.
			\end{thm}
		
	\section{Eigenvalues and Eigenvectors}
		\subsection{Characteristic Polynomial}
		
		\subsection{}
		
	
	\section{Special Matrices}
		\subsection{Diagonal Matrices}
		A \emph{diagonal matrix} is a square matrix in which all entries off the main diagonal are zero. They can be represented in the following form:
		
		\begin{displaymath}
		D=
		\begin{bmatrix}
		d_1 & 0 & \cdots & 0 \\
		0 & d_2 & \cdots & 0 \\
		\vdots & \vdots &        & \vdots \\
		0 & 0 & \cdots & d_n \\
		\end{bmatrix}
		\end{displaymath}
		
		A diagonal matrix is invertible iff all of its diagonal entries are nonzero, and its inverse is:
		
		\begin{displaymath}
		D^{-1}=
		\begin{bmatrix}
		d_1^{-1} & 0 & \cdots & 0 \\
		0 & d_2^{-1} & \cdots & 0 \\
		\vdots & \vdots &        & \vdots \\
		0 & 0 & \cdots & d_n^{-1} \\
		\end{bmatrix}
		\end{displaymath}
		
		It is easy to calculate powers of diagonal matrices. More specifically,
		
		\begin{displaymath}
		D^{k}=
		\begin{bmatrix}
		d_1^{k} & 0 & \cdots & 0 \\
		0 & d_2^{k} & \cdots & 0 \\
		\vdots & \vdots &        & \vdots \\
		0 & 0 & \cdots & d_n^{k} \\
		\end{bmatrix}
		\end{displaymath}
		
		\subsection{Triangular Matrices}
		A \emph{lower triangular} matrix is a matrix in which all the entries above the main diagonal are zero; an \emph{upper triangular} matrix is a matrix in which all the entries below the main diagonal are zero. Either of they are called \emph{triangular}. They can be represented in the following form:
		
		\begin{displaymath}
		L=
			\begin{bmatrix}
				l_{11}     & 0          & \cdots & 0              & 0      \\
				l_{21}     & l_{22}     & \cdots & 0              & 0      \\
				\vdots     & \vdots     &        & \vdots         & \vdots \\
				l_{(n-1)1} & l_{(n-1)2} & \cdots & l_{(n-1)(n-1)} & 0      \\
				l_{n1}     & l_{n2}     & \cdots & l_{n(n-1)}     & l_{nn} \\
			\end{bmatrix}
		, U=
			\begin{bmatrix}
				u_{11} & u_{12} & \cdots & u_{1(n-1)}     & u_{1n}     \\
				0      & u_{22} & \cdots & u_{2(n-1)}     & u_{2n}     \\
				\vdots & \vdots &        & \vdots         & \vdots     \\
				0      & 0      & \cdots & u_{(n-1)(n-1)} & u_{(n-1)n} \\
				0      & 0      & \cdots & 0              & u_{nn}     \\
			\end{bmatrix}
		\end{displaymath}
		
		\begin{thm}
			\begin{enumerate}
				\item The transpose of a lower triangular matrix is upper triangular, and vice versa.
				\item The product of lower triangular matrices is lower triangular, and same for the upper triangular matrices.
				\item A triangular matrix is invertible iff its diagonal entries are all nonzero.
				\item The inverse of an invertible lower triangular matrix is lower triangular, and same for the invertible upper triangular matrices.
			\end{enumerate}
		\end{thm}
		
		\subsection{Symmetric Matrices}
		A \emph{symmetric} matrix is a square matrix such that $S=S^T$. Specifically, $S$ is symmetric iff $\forall 1 \le i,j \le n, S_{ij}=S_{ji}$.
		
		It is important to note that for two symmetric matrices $A$ and $B$, $(AB)^T=B^TA^T=BA$, and therefore their product is not guaranteed to be symmetric unless $AB=BA$, that is, $A$ and $B$ commute.
		
		\begin{thm}
			The product of two symmetric matrices is symmetric iff the matrices commute.
		\end{thm}
		
		In general, a symmetric matrix may not be invertible. However if they are, the following theorem shows an interesting fact:
		
		\begin{thm}
			If $A$ is an invertible symmetric matrix, then $A^{-1}$ is symmetric.
		\end{thm}
	
	\section{Solving Linear Equations}
	We come to this final section, the ultimate target of linear algebra: solving a system of linear equations.
		\subsection{Linear Equations to Matrices}
		A finite set of linear equations is called a \emph{system of linear equations}, or more briefly, a \emph{linear system}. The variables are called \emph{unknowns}.
		
		\begin{center}
			\begin{tabular}{ccccccccc}
				$a_{11}x_1$ & $+$ & $a_{12}x_2$ & $+$ & $\cdots$ & $+$ & $a_{1n}x_n$ & $=$ & $b_1$    \\
				$a_{21}x_1$ & $+$ & $a_{22}x_2$ & $+$ & $\cdots$ & $+$ & $a_{2n}x_n$ & $=$ & $b_2$    \\
				$\vdots$    &     & $\vdots$   &     &          &     & $\vdots$    &     & $\vdots$ \\
				$a_{m1}x_1$ & $+$ & $a_{m2}x_2$ & $+$ & $\cdots$ & $+$ & $a_{mn}x_n$ & $=$ & $b_m$   
			\end{tabular}
		\end{center}
		
		A \emph{solution} of a linear system in $x_1,x_2,\dots,x_n$ is a sequence of $n$ numbers $s_1,s_2,\dots,s_n$ for which the substitution $x_i=s_i$ makes each equation a true statement.
		
		We say that a linear system is \emph{consistent} if it has at least one solution and \emph{inconsistent} if it has no solutions.
		
		\begin{thm}
			A system of linear equations has zero, one, or infinitely many solutions. There are no possibilities.
		\end{thm}
		
		If a linear system has infinitely many solutions, then a set of parametric equations from which all solutions can be obtained by assigning numerical values to the parameters is called a \emph{general solution} of the system.
		
		If all constant terms are zero, that is, $\forall i, b_i=0$, it is said to be \emph{homogeneous}. A homogeneous system of linear equations always is consistent since it has $\forall i, x_i=0$ as its solution: this is called the \emph{trivial solution}. If there are other solutions, they are called the \emph{nontrivial solution}.
		
		The system of linear equations above can be represented in a matrix multiplication as shown below:
		
		\begin{displaymath}
			\begin{bmatrix}
				a_{11} & a_{12} & \cdots & a_{1n} \\
				a_{21} & a_{22} & \cdots & a_{2n} \\
				\vdots & \vdots &        & \vdots \\
				a_{m1} & a_{m2} & \cdots & a_{mn} \\
			\end{bmatrix}
			\begin{bmatrix}
				x_1 \\ x_2 \\ \vdots \\ x_n
			\end{bmatrix}
			=
			\begin{bmatrix}
				b_1 \\ b_2 \\ \vdots \\ b_m
			\end{bmatrix}
		\end{displaymath}
		
		By designating the three matrices $A$, $\bf{x}$ and $\bf{b}$ respectively, we can say that $A\bf{x}=\bf{b}$. In this equation, $A$ is called the \emph{coefficient matrix} of the system.
		
		The \emph{augmented matrix} for the system is obtained by adjoining $\bf{b}$ to $A$ as the last column as follows:
		
		\begin{displaymath}
			\left[\begin{array}{cccc|c}
				a_{11} & a_{12} & \cdots & a_{1n} & b_1\\
				a_{21} & a_{22} & \cdots & a_{2n} & b_2\\
				\vdots & \vdots &        & \vdots & \vdots\\
				a_{m1} & a_{m2} & \cdots & a_{mn} & b_m\\
			\end{array}\right]
		\end{displaymath}
		
		Note the correspondence between basic algebraic operations on a given set of linear systems and elementary row operations on the augmented matrix of the said systems. In the order in the definition [\ref{def_elementary_row_operations}], the correspondences are:
		
		\begin{enumerate}
			\item Multiply an equation through by a nonzero constant
			\item Interchange two equations	
			\item Add a constant times one equation to another.
		\end{enumerate}
		
		By applying elementary row operations to the augmented matrix, we can get to the point where the augmented matrix is reduced to its reduced row echelon form. The variables corresponding to the leading 1's in the augmented matrix is called the \emph{leading variables}. The remaining variables are called \emph{free variables}.
		
		There is an important theorem regarding the number of free variables and homogeneous systems:
		
		\begin{thm}[Free Variable Theorem for Homogeneous Systems]
			If a homogeneous linear system has $n$ unknowns, and if the rref of its augmented matrix has $r$ nonzero rows, then the system has $n-r$ free variables.
		\end{thm}
		
		\begin{coro}
			A homogeneous linear system with more unknowns than equations has infinitely many solutions.
		\end{coro}
		
		In the following sections on finding solutions or parametric equation for solutions where it applies, the coefficient matrix will be noted as $A$, the vector of variables will be noted as $\bf{x}$ and the variables as $x_1, x_2, \dots, x_n$, and the vector for the constants as $\bf{b}$.
		
		\subsection{Method of Inverses}
		
		This can be used iff $A$ is an invertible matrix.
		
		Find the inverse of $A$, $A^{-1}$.
		The only possible solution is $\bf{x}=A^{-1}\bf{b}$.
		
		\subsection{Method of RREF}\label{mthd_rref}
		
		This can be used for any matrix $A$.
		
		\begin{enumerate}
			\item Reduce the augmented matrix $\left[A|\bf{b}\right]$ to its RREF $\left[R|\bf{c}\right]$
			\item See if $R$ has a zero row. If any of the value of $\bf{c}$ corresponding to the zero row is nonzero, the system is inconsistent.
			\item Exchange the free variables with parametric variables.
			\item Transpose the free variables to RHS so the leading variables(the pivots) are the only ones left on the LHS.
			\item The resulting expressions are the parametric equation for solutions.
		\end{enumerate}
		
		\subsection{Method of Particular and Special Special Solutions}
			
			This can be used for any matrix $A$.
			
			This method is extremely similar to Method of RREF[\ref{mthd_rref}].
			
			In this method, we first find the nullspace of $A$.
			
			\begin{thm}
				If $A$ is an $m \times n$ matrix, then the solution set of the homogeneous linear system $A\bf{x}=\bf{0}$ consists of all vectors in $\mathbb{R}^n$ that are orthogonal to every row vector of $A$.
			\end{thm}
			
			\begin{thm}
				The general solution of a consistent linear system $A\bf{x}=\bf{b}$ can be obtained by adding any specific solution of $A\bf{x}=\bf{b}$ to the general solution of $A\bf{x}=\bf{0}$.
			\end{thm}
			
			The theorem above indicates that we need to find the nullspace of $A$ along with a specific solution of $A\bf{x}=\bf{b}$ to find the whole, general solution of $A\bf{x}=\bf{b}$.
			
			Using Gauss-Jordan Elimination[\ref{mthd_gauss_jordan_elim}], we reduce the augmented matrix $\left[A|\bf{b}\right]$ to its rref, and detect if there are any inconsistencies. This corresponds to the first step on Method of RREF.
			
			First, we find the nullspace for $A$. We get the basis vectors from the rref of $A$. This corresponds to the second and third steps on Method of RREF.
			
			Next, we find the specific solution of $A\bf{x}=\bf{b}$. From the rref of $\left[A|\bf{b}\right]$ say $\left[R|\bf{c}\right]$, solve the equation by setting all free variables to 0. In doing so, since all leading variables have coefficient 1, the values of $c$ immediately correspond to the specific solution of the leading variables. This process is therefore almost automatic.
			
			Now we have the nullspace of $A$ and the specific solution of $A\bf{x}=\bf{b}$; add those two together to gain the whole solution. This corresponds to the fourth and final steps on Method of RREF.
		
	
	Before ending this chapter, we summarize this chapter by gathering all the facts on invertible matrices, written in the appendix[\ref{equiv_invert}].
\end{document}

%Introduction to Linear Algebra 4th ed, Gilbert Strang
		
		\chapter{Calculus}
		    \input{./chapters/mathematics/calculus.tex}
		
		\chapter{Statistics}
		    \input{./chapters/mathematics/statistics.tex}
			
		\chapter{From $\mathbb{N}$ to $\mathbb{R}$}
		    \documentclass{report}

\begin{document}
	\section{$\mathbb{N}$: The set of Natural Numbers}
		\subsection{Construction of $\mathbb{N}$}
		We start from the Axioms of Set[\ref{axiom_empty_set},\ref{axiom_extensionality},\ref{axiom_pairing},\ref{axiom_union_simple},\ref{axiom_union_full},\ref{axiom_subset}], the definition of power set[\ref{def_powerset}], the definition of equivalence relation and class[\ref{def_equiv_rel},\ref{def_equivalence_class}] and the following definitions:
		
		\begin{defn}[Successor] \label{defn_successor}
			For any set $x$, the \emph{successor of $x$}, denoted $\sigma(x)$, is defined as the following set:
			\begin{displaymath}
			\sigma(x)=x \cup \{x\}
			\end{displaymath}
		\end{defn}
		
		Let us define $0=\emptyset$, $1=\sigma(\emptyset)=\sigma(0)$. Using the definition of successors, and following the pattern, $2=\sigma(1)$, $3=\sigma(2)$, and so on. Basically we can make any finite number using the definition of successor and the Axioms of Set, but actually getting all of the natural numbers at once(or any infinitely large set, since only the empty set is guaranteed to exist by the axioms) is not possible with our axioms. We define the concept of Inductive Sets and make another Axiom for this purpose:
		
		\begin{defn}[Inductive Set] \label{defn_inductive_set}
			A set $A$ is called \emph{inductive} if it satisfies the following two properties:
			\begin{itemize}
				\item $\emptyset \in A$
				\item $(x \in A) \Rightarrow (\sigma(x) \in A)$
			\end{itemize}
		\end{defn}
		
		\begin{axiom}[Axiom of Infinity] \label{axiom_infinity}
			There is an inductive set, that is:
			\begin{displaymath}
			\exists A (\emptyset \in A) \wedge (\forall x \in A, \sigma(x) \in A)
			\end{displaymath}
		\end{axiom}
		
		\begin{thm}
			Take any two inductive sets, $S$ and $T$. Then, $S \cap T$ is also an inductive set.
		\end{thm}
		
		\begin{proof}
			Let $U=S \cap T$.
			\begin{enumerate}
				\item $\emptyset \in U$
				\subitem $\emptyset \in S$ and $\emptyset \in T$ since $S$ and $T$ are both inductive.
				\item $(x \in U) \Rightarrow (\sigma(x) \in U)$
				\subitem $\forall x \in U, (x \in S) \wedge (x \in T)$.\\
				Since $S$ and $T$ are both inductive, $(\sigma(x) \in S) \wedge (\sigma(x) \in T)$\\
				Therefore $\sigma(x) \in U$.
			\end{enumerate} 
			Therefore $U$ is inductive.
		\end{proof}
		
		\begin{coro}
			An intersection of any number of inductive sets is inductive.
		\end{coro}
		
		\begin{thm}
			For any inductive set $S$, define $N_S$ as follows:
			\begin{displaymath}
			N_S=\bigcap_{\substack{A \subseteq S\\A \text{ is inductive}}}A
			\end{displaymath}
			Take any two inductive sets, $S$ and $T$. Then $N_S=N_T$.
		\end{thm}
		
		\begin{proof}
			Suppose not; WLOG, $\exists x$ such that $x \in N_S$ and $x \notin N_T$.\\
			Let $X=N_S \cap N_T$. Then $X$ is inductive, $X \subset N_S$, and $x \notin X$.\\
			Since by the definition of $N_S$, $N_S=X \cap N_S$, but $x \notin X \cap N_S$ hence the RHS and the LHS are different.\\
			Therefore the assumption is wrong; therefore $N_S=N_T$.
		\end{proof}
		
		Using this theorem, we can finally define the set of natural numbers:
		\begin{defn}[The Set $\mathbb(N)$ of natural numbers] \label{def_N}
			Take any inductive set $S$, and let
			\begin{displaymath}
			N=\bigcap_{\substack{A \subseteq S\\A \text{ is inductive}}}A
			\end{displaymath}
			This set is the natural numbers, which we denote as $\mathbb{N}$.
		\end{defn}
		
		\subsection{Operations on $\mathbb{N}$}
		We now define two operations on $\mathbb{N}$, addition($+$) and multiplication($\cdot$).
		
		\begin{defn}[Addition and Multiplication on $\mathbb{N}$] \label{def_add_mult_N}
			The operation of addition, denoted by $+$, is defined by following two recursive rules:
			\begin{enumerate}
				\item $\forall n \in \mathbb{N}, n+0=n$
				\item $\forall n,m \in \mathbb{N}, n+\sigma(m)=\sigma(n+m)$
			\end{enumerate}
			Similarly the operation of multiplication, denoted by $\cdot$, is defined by following two recursive rules:
			\begin{enumerate}
				\item $\forall n \in \mathbb{N}, n \cdot 0=0$
				\item $\forall n,m \in \mathbb{N}, n \cdot \sigma(m)=n \cdot m+n$
			\end{enumerate}
		\end{defn}
		
		\begin{lemma}[Operations on $0$] \label{thm_n_op_on_0}
			$\forall x \in \mathbb{N}$
			\begin{itemize}
				\item $x+0=0+x$
				\item $x \cdot 0=0 \cdot x$
			\end{itemize}
		\end{lemma}
		
		\begin{prop}[Properties of $+$ and $\cdot$] \label{thm_property_operation_N}
			$\forall x,y,z \in \mathbb{N}$,
			\begin{itemize}
				\item \textbf{Associativity of Addition} $x+(y+z)=(x+y)+z$
				\item \textbf{Commutativity of Addition} $x+y=y+x$
				\item \textbf{Associativity of Multiplication} $x \cdot (y \cdot z)=(x \cdot y) \cdot z$
				\item \textbf{Commutativity of Multiplication} $x \cdot y=y \cdot x$
				\item \textbf{Distributive Law} $x \cdot (y+z)=x \cdot y+x \cdot z$
				\item \textbf{Cancellation Law for Addition} $x+z=y+z \Rightarrow x=y$
			\end{itemize}
		\end{prop}
		
		\subsection{Ordering on $\mathbb{N}$}
		\begin{defn}[Ordering on $\mathbb{N}$] \label{def_order_N}
			For $n,m \in \mathbb{N}$, we say that $n$ is less than $m$, written $n \ge m$, if there exists a $k \in \mathbb{N}$ such that $m=n+k$. We also write $n<m$ if $k \ne 0$.
		\end{defn}
		
		\begin{thm}
			$(N,<)$ is an ordered set[\ref{def_ordered_set}].
		\end{thm}
		
		\begin{prop} \label{thm_property_order_N}
			The followings are true:
			\begin{itemize}
				\item If $n \ne 0$, then $0<n$.
				\item Let $x,y,z \in \mathbb{N}$. Then the followings are true:
				\begin{itemize}
					\item $(x \le y) \wedge (y<z) \Rightarrow (x<z)$
					\item $(x<y) \wedge (y \le z) \Rightarrow (x<z)$
					\item $(x \le y) \wedge (y \le z) \Rightarrow (x \le z)$
					\item $(x<y) \Rightarrow (x+z<y+z)$
					\item $(x<y) \Rightarrow (xz<yz)$
				\end{itemize}
				\item $\forall n \in \mathbb{N}, n \ne n+1$
				\item $\forall n,k \in \mathbb{N}, k \ne 0, n \ne n+k$
			\end{itemize}
		\end{prop}
		
		\begin{defn}[Least Element] \label{def_least_element}
			Let $S \subset \mathbb{N}$. An element $n \in S$ is called a \emph{least element} if $\forall m \in S, n \le m$
		\end{defn}
		
		\begin{prop}[Uniqueness of the Least Element] \label{thm_unique_least_element}
			Let $S \subset \mathbb{N}$. Then if $S$ has a least element, then it is unique.
		\end{prop}
		
		\begin{thm}[Well-Ordering Property] \label{thm_well_ordering_property}
			Let $S$ be a nonempty subset of $\mathbb{N}$. Then $S$ has a least element.
		\end{thm}
		
		\begin{note}
			The well-ordering property states that the set of natural numbers $\mathbb{N}$ has the greatest lower bound property[\ref{def_greatest_lower_bound_property}] and thereby theorem \ref{thm_glb_lub_property}, has the least upper bound property[\ref{def_least_upper_bound_property}].
		\end{note}
		
		\subsection{Properties of $\mathbb{N}$}
		Many of the mathematics book defines the set of Natural Numbers as the set satisfying the \emph{Peano Axioms}.
		\begin{prop}[Peano Axioms] \label{peano_axioms} \label{thm_property_N}
			\begin{enumerate}
				\item[]
				\item $0$, which we defined as the empty set $\emptyset$, is a natural number.
				\item There exist a distinguished set map $\sigma: \mathbb{N} \rightarrow \mathbb{N}$
				\item $\sigma$ is injective
				\item There does not exist an element $n \in \mathbb{N}$ such that $\sigma(n)=0$
				\item (Principle of Induction) If $S \in N$ is inductive, then $S=N$.
			\end{enumerate}
		\end{prop}
		
		\begin{prop}
			Suppose that $a$ is a natural number, and that $b \in a$. Then $b \subseteq a$, $a \nsubseteq b$.
		\end{prop}
		
		\begin{prop}
			For any two natural numbers $a,b\in \mathbb{N}$, if $\sigma(a)=\sigma(b)$, then $a=b$.
		\end{prop}
		
		\begin{lemma}
			If $n \in \mathbb{N}$ and $n \ne 0$, then there exists $m \in \mathbb{N}$ such that $\sigma(m)=n$.
		\end{lemma}
	
	\section{$\mathbb{Z}$: The set of Integers}
		\subsection{Construction of $\mathbb{Z}$}
		We now have the set of natural numbers, and starting there, we construct the set of integers.
		
		\begin{prop} \label{def_Z_equiv_class}
			Define a relation $\equiv$ on $\mathbb{N} \times \mathbb{N}$ by $(a,b) \equiv (c,d)$ iff $a+d=b+c$. This relation is an equivalence relation on $\mathbb{N} \times \mathbb{N}$.
		\end{prop}
	
		Let $\mathbb{Z}$ be the set of equivalence classes under this relation, and the equivalence class containing $(a,b)$ be denoted by $[a,b]$.
		
		\subsection{Operations on $\mathbb{Z}$}
		\begin{defn}[Addition and Multiplication on $\mathbb{Z}$] \label{def_add_mult_Z}
			Addition and multiplication on $\mathbb{Z}$ are defined by:
			\begin{itemize}
				\item $[a,b]+[c,d]=[a+c,b+d]$
				\item $[a,b] \cdot [c,d]=[ac+bd,ad+bc]$
			\end{itemize}
		\end{defn}
		
		\begin{defn}[Subtraction on $\mathbb{Z}$] \label{def_sub_Z}
			Subtraction on $\mathbb{Z}$ is defined by:
			\begin{displaymath}
				[a,b]-[c,d]=[a,b]+[d,c]
			\end{displaymath}
		\end{defn}
	
		\subsection{Ordering on $\mathbb{Z}$}

			\begin{defn}[Ordering on $\mathbb{Z}$] \label{def_order_Z}
				Let $[a,b],[c,d] \in \mathbb{Z}$. $[a,b]<[c,d]$ iff $a+d<b+c$.
			\end{defn}

\begin{comment}%todo: find out what the positive set is
			\begin{prop}
				The subset $\mathbb{N}=\{[0,n]|n \text{ is a natural number}\}$ is a positive set in $\mathbb{Z}$.
			\end{prop}
		
			\begin{proof}
				$\mathbb{N}$ is closed under addition and multiplication by definition.\\
				Let $[a,b] \in  \mathbb{Z}$. Since $a,b \in \mathbb{N}$ and $\mathbb{N}$ is well-ordered, either $a<b$, $a=b$, or $b<a$. So either $[a,b]$ is $[0,b-a]$, $[0,0]$, or $[a-b,0]$.\\
				Thus either $[a,b]\in\mathbb{N}$, $[a,b]=0$, or $-[a,b]=[b,a]\in\mathbb{n}$.
			\end{proof}
\end{comment}
	
		\subsection{Property of $\mathbb{Z}$}
		
			\begin{thm}[Arithmetic Properties of $\mathbb{Z}$] \label{thm_property_Z}
				\begin{enumerate}
					\item[]
					\item Addition and multiplication are well-defined.
					\item Addition and multiplication have identity elements $[n,n]$ and $[n,n+1]$, respectively.
					\item Addition and multiplication are commutative and associative.
					\item The distributive law holds.
					\item Each element $[a,b]$ has an additive inverse $[b,a]$.
				\end{enumerate}
			\end{thm}
	
			We can treat $\mathbb{N}$ to be a subset of $\mathbb{Z}$ by identifying the number $n$ with the class $[0,n]$. Since $[0,a]+[0,b]=[0,a+b]$ and $[0,a] \cdot [0,b]=[0,ab]$, these operations mirror the corresponding operation in $\mathbb{N}$.
			
			Given $n \in \mathbb{N}$, we write $-n$ for $[n,0]$, $0$ for $[n,n]$, and $1$ for $[n,n+1]$. By the fifth arithmetic property of $\mathbb{Z}$[\ref{thm_property_Z}], this defines $-n$ to be the additive inverse of $n$. We also use the minus sign for subtraction; it is therefore natural to write $[a,b]$ as $b-a$.
			
		\begin{prop}
			For $a,b \in \mathbb{N}$, let $-b$, $a$, and $b$ be defined in $\mathbb{Z}$ as above. Then
			\begin{displaymath}
				a-b=a+(-b) \text{   and   } -(-b)=b
			\end{displaymath}
		\end{prop}

	\section{$\mathbb{Q}$: The set of Rational Numbers}
		We construct the set of rational numbers from the set of integers as follows:
		\subsection{Construction of $\mathbb{Q}$}
		
			\begin{prop}\label{def_Q_equiv_class}
				Define a relation $\equiv$ on $\mathbb{Z} \times (\mathbb{Z} \textbackslash \{0\})$ by $(a,b)\equiv(c,d)$ iff $ad=bc$. This relation is an equivalence relation on $\mathbb{Z} \times (\mathbb{Z} \textbackslash \{0\})$.
			\end{prop}
		
			Let $\mathbb{Q}$ be the set of equivalence classes under this relation, and the equivalence class containing $(a,b)$ is denoted by $a/b$ or $\frac{a}{b}$, and $\frac{a}{b}=\frac{c}{d}$ mean that $(a,b)$ and $(c,d)$ belong to the same equivalence class. Especially we write $0$ and $1$ to denote $\frac{0}{1}$ and $\frac{1}{1}$, respectively.
		
		\subsection{Operations on $\mathbb{Q}$}
		
			\begin{defn}[Addition and Multiplication on $\mathbb{Q}$] \label{def_add_mult_Q}
				The \emph{sum} and \emph{product} of $\frac{a}{b}, \frac{c}{d} \in \mathbb{Q}$ are defined by
				\begin{displaymath}
					\frac{a}{b}+\frac{c}{d}=\frac{ad+bc}{bd} \text{   and   } \frac{a}{b}\frac{c}{d}=\frac{ac}{bd}
				\end{displaymath}
			\end{defn}
		
			\begin{defn}[Subtraction on $\mathbb{Q}$] \label{def_sub_Q}
				Subtraction on $\mathbb{Z}$ is defined by:
				\begin{displaymath}
					\frac{a}{b}-\frac{c}{d}=\frac{ad-bc}{bd}
				\end{displaymath}
			\end{defn}
			
			\begin{defn}[Division on $\mathbb{Q}$] \label{def_div_Q}
				Division on $\mathbb{Z}$ is defined by:
				\begin{displaymath}
					\frac{a}{b}\div\frac{c}{d}=\frac{ad}{bc}
				\end{displaymath}
			\end{defn}
		
		\subsection{Ordering on $\mathbb{Q}$}
		
			\begin{defn}[Ordering on $\mathbb{Q}$] \label{def_order_Q}
				Let $\frac{a}{b},\frac{c}{d} \in \mathbb{Q}$. $\frac{a}{b}<\frac{c}{d}$ iff $(bd>0 \wedge ad<bc) \vee (bd<0 \wedge ad>bc)$.
			\end{defn}
			
\begin{comment}%todo: find out what the positive set is
			\begin{prop}
				The subset $P=\{\frac{a}{b}\in\mathbb{Q}|ab>0}$ is a positive set in $\mathbb{Q}$.
			\end{prop}
\end{comment}
		
		\subsection{Property of $\mathbb{Q}$}
		
			\begin{thm}[Arithmetic Properties of $\mathbb{Q}$] \label{thm_property_Q}
				\begin{enumerate}
					\item[]
					\item Addition and multiplication are well-defined.
					\item Addition and multiplication have identity elements $0$ and $1$, respectively.
					\item Addition and multiplication are commutative and associative.
					\item The distributive law holds.
				\end{enumerate}
			\end{thm}
		
			\begin{thm}
				$(\mathbb{Q},+,\cdot)$ forms an ordered field.
			\end{thm}
		
	\section{$\mathbb{R}$: The set of Real Numbers}
		\subsection{Construction of $\mathbb{R}$}
			One simple way to construct $\mathbb{R}$ is by proving the following theorem:
			
			\begin{thm}[Existence of $\mathbb{R}$] \label{thm_existence_real_number}
				There exists an ordered field $\mathbb{R}$ containing $\mathbb{Q}$ as a subfield which has the least-upper-bound property.
			\end{thm}
		
			But where's the fun in that? We will be constructing the field of real numbers using Cauchy sequences[\ref{def_cauchy_sequence}], starting with the following proposition:
		
			\begin{thm} \label{def_R_equiv_class}
				Define a relation $\equiv$ on the set $S$ of Cauchy sequences of rational numbers as follows:
				\begin{displaymath}
					\{a_n\} \equiv \{b_n\} \text{   iff   } (a_n-b_n)\rightarrow 0
				\end{displaymath}
				This relation is an equivalence relation.
			\end{thm}
		
			Now let us define $\mathbb{R}$ as the set of equivalence classes of $S$ under the relation $\equiv$.

		\subsection{Operations on $\mathbb{R}$}
		
			Before the definition of operations on $\mathbb{R}$, we need to find out whether if the Cauchy sequences of rational numbers are closed under addition and multiplication, and it turns out they do, as stated in the following proposition:
		
			\begin{prop}
				The set $S$ of Cauchy sequences of rational numbers is closed under addition, multiplication, and scalar multiplication, that is:
				\begin{enumerate}
					\item If $\{a_n\}\in S$ and $\{b_n\}\in S$, then $\{a_n+b_n\}\in S$
					\item If $\{a_n\}\in S$ and $\{b_n\}\in S$, then $\{a_nb_n\}\in S$
					\item If $\{a_n\}\in S$ and $c \in \mathbb{Q}$, then $\{ca_n\}\in S$
				\end{enumerate}
			\end{prop}
		
			We can finally go on to defining the operations on $\mathbb{R}$.
		
			\begin{defn}[Addition and Multiplication on $\mathbb{R}$] \label{def_add_mult_R}
				Let $\{a_n\}$ and $\{b_n\}$ be sequences contained in the real numbers $\alpha$, $\beta$, respectively. Then the \emph{sum} and \emph{product} of $\alpha$ and $\beta$ are defined by:
				\begin{displaymath}
					\alpha+\beta=\{a_n+b_n\} \text{   and   } \alpha\beta=\{a_n b_n\}
				\end{displaymath}
			\end{defn}
		
			We can define subtraction and division on $\mathbb{R}$ similar to addition and multiplication, by term-by-term calculation on each term of the Cauchy sequence.
		
		\subsection{Ordering on $\mathbb{R}$}
		
			\begin{defn}[Ordering on $\mathbb{R}$] \label{def_order_R}
				Let $\alpha=\{a_n\},\beta=\{b_n\} \in \mathbb{R}$. $\alpha<\beta$ iff $\exists N \in \mathbb{N}, \forall n \ge N, a_n<b_n$.
			\end{defn}
		
		\subsection{Property of $\mathbb{R}$}
		
			\begin{thm}[Arithmetic Properties of $\mathbb{R}$] \label{thm_property_R}
				\begin{enumerate}
					\item[]
					\item Addition and multiplication are well-defined.
					\item Addition and multiplication have identity elements $\{0\}$ and $\{1\}$, respectively.
					\item Addition and multiplication are commutative and associative.
					\item The distributive law holds.
					\item Each element $\{a_n\}$ has an additive inverse $\{-a_n\}$.
				\end{enumerate}
			\end{thm}
		
			\begin{thm}
				$(\mathbb{R},+,\cdot)$ forms an ordered field.
			\end{thm}

			We now define an extension to $\mathbb{R}$ as follows:

			\begin{defn}[Extended Real Number System] \label{def_extended_real_number_system}
				The \emph{extended real number system}, denoted $\mathbb{R}^+$, $[-\infty,\infty]$, or $\mathbb{R} \cup \{-\infty,\infty\}$, consists of the real field $\mathbb{R}$ and two symbols, $+\infty$ and $-\infty$. We preserve the original order in $\mathbb{R}$, and define $\forall x \in \mathbb{R}$,
				\begin{displaymath}
				-\infty<x<\infty
				\end{displaymath}
			\end{defn}
			
			\begin{remark} \label{remark_extended_real_number_system_not_field}
				The extended real number system does not form a field.
			\end{remark}

	\section{$\mathbb{C}$: The set of Complex Numbers}
			We construct the set of complex numbers from $\mathbb{R}$. Unlike the previous constructions, we do not construct it using equivalence class. Instead the construction is done by considering the quotient ring of polynomial ring over $\mathbb{R}$ modulo $i^2+1$.
			
			\begin{defn} \label{def_C_quotient_ring}
				Complex number is defined as the quotient ring $\mathbb{R}[i]/(i^2+1)$, with operations defined as normal.
			\end{defn}
		
			\begin{thm}
				$(\mathbb{C},+,\cdot)$ forms a field.
			\end{thm}
		
\end{document}
	
	\part{Advanced Topics}
	    
	    \chapter{Abstract Algebra}
	        %\documentclass{report}

\begin{document}
    \section{Group Basics}
    
    \subsection{Groups}
    
    The first thing we would encounter in abstract algebra is a group... but you already encountered it. Refer to the chapter "Algebraic Structures" for the definition of a group and an abelian group.
    
    If the context is obvious, we will skip $\cdot$ and write $ab$ instead of $a \cdot b$. The identity of $G$ will be denoted $e$ or $1$.
    
    \begin{defn} \label{def_group_power}
        The product of $n$ occurrences of $x$ is denoted $x^n$. The product of $n$ occurrences of $x^{-1}$ is denoted $x^{-n}$. Also $x^0 = 1$.
    \end{defn}
    
    \begin{thm} \label{thm_group_basics}
        If $G$ is a group, and $a,b,c \in G$, then \begin{enumerate}
            \item the identity of $G$ is unique.
            \item the inverse $a^{-1}$ is unique.
            \item $(a^{-1})^{-1}=a$.
            \item $(ab)^{-1} = b^{-1}a^{-1}$.
            \item if $ab=ac$, then $b=c$. Also, if $ba=ca$, then $b=c$.
            \item For $n,m \in \mathbb{Z}$, $x^nx^m = x^{n+m}$ and $(x^n)^{-1} = x^{-n}.$
        \end{enumerate}
    \end{thm}
    
    \begin{proof}
        \begin{enumerate} \item[]
        \item Let $e_1$ and $e_2$ be the identities of $G$. Then $e_1e_2=e_2e_1=e_1$, and $e_2e_1=e_1e_2=e_2$, from the definition of the identity. Therefore $e_1=e_2$.
        \item Let $b_1$ and $b_2$ the inverses of $a$. Then $b_1 = b_1(ab_2) = (b_1a)b_2 = b_2$.
        \item The definition of an inverse shows that $a$ is an inverse of $a^{-1}$. From (ii), such an inverse is unique.
        \item $(ab)b^{-1}a^{-1} = a(bb^{-1})a^{-1} = aa^{-1} = e$. Similarly $b^{-1}a^{-1}(ab) = e$. From (ii), the inverse of $ab$ is unique.
        \item $ab=ac \implies a^{-1}ab=a^{-1}ac \implies b=c$. Similar argument for $ba=ca$.
        \item TODO
        \end{enumerate}
    \end{proof}
    
    \begin{defn} \label{def_group_misc}
        Let $G$ be a group. \begin{itemize}
        \item $a,b \in G$ \emph{commute} if $ab=ba$.
        \item The \emph{order} of $x \in G$, denoted $|x|$, is the smallest positive integer $n$ such that $x^n=1$. If no such $n$ exists, then $|x|=\infty$.
        \item The \emph{order} of $G$, denoted $|G|$, is the cardinality of $G$ as a set.
    \end{itemize} \end{defn}
    
    \begin{defn} \label{def_group_example} \begin{itemize}
        \item[]
        \item TODO: Z/nZ
        \item TODO: Sn
    \end{itemize} \end{defn}
    
    \subsection{Isomorphism}
    
    Now, we want to tell whether two groups are "same," in the sense that there is a bijection between them preserving the relations.
    
    \begin{defn}[Homomorphisms and Isomorphisms] \label{def_group_homomorphism}
        Let $(G,\star)$, $(H,\diamond)$ be two groups. Then a map $\varphi: G \to H$ is a \emph{homomorphism} if for all $x,y \in G$, $\varphi(x \star y)=\varphi(x) \diamond \varphi(y)$. An \emph{isomorphism} is a bijective homomorphism. If there is an isomorphism between $G$ and $H$, we say they are \emph{isomorphic} and denote $G \cong H$.
    \end{defn}
    
    We may skip $\star$ and $\diamond$ here too if the context is clear, but you have to understand which operations are used at each positions.
    
    \begin{thm} \label{thm_isomorphic_groups}
        If $(G,\star)$ and $(H,\diamond)$ are isomorphic, with the isomorphism $\varphi$, then \begin{enumerate}
            \item $|G| = |H|$.
            \item $G$ is abelian iff $H$ is abelian.
            \item For any $x \in G$, $|x| = |\varphi(x)|$.
        \end{enumerate}
    \end{thm}
    
    \begin{proof}
    (1) $\varphi$ is bijective.
    
    (2) Suppose $G$ is abelian. Take $c,d \in H$. Since $\varphi$ is surjective, there are $a,b \in G$ such that $\varphi(a) = c$ and $\varphi(b) = d$. Then $cd = \varphi(a)\varphi(b) = \varphi(ab) = \varphi(ba) = \varphi(b)\varphi(a) = dc$. Therefore $H$ is abelian.
    
    Suppose $H$ is abelian. Take $a,b \in G$. Then $\varphi(ab) = \varphi(a)\varphi(b) = \varphi(b)\varphi(a) = \varphi(ba)$. Since $\varphi$ is injective, $ab = ba$. Therefore $G$ is abelian.
    
    (3) TODO
    \end{proof}
    
    \subsection{Group Actions}
    
    \begin{defn}[Group Action] \label{def_group_action}
        A \emph{group action} of a group $G$ on a set $A$ is a map $G \times A \to A$, mapping $g \times a$ to $g \cdot a$, such that for all $g_1,g_2 \in G$ and $a \in A$, \begin{enumerate}
            \item $g_1 \cdot (g_2 \cdot a) = (g_1g_2) \cdot a$, and
            \item $1 \cdot a = a$.
        \end{enumerate}
        We say that $G$ \emph{acts} on $A$.
    \end{defn}
    
    Again, we may skip $\cdot$.
    
    TODO: permutation representation
    
    \subsection{Subgroups}
    
    \begin{defn}[Subgroups] \label{def_subgroup}
        Let $G$ be a group. A subset $H$ of $G$ is a \emph{subgroup} of $G$ if $H$ is nonempty, and for all $x,y \in H$, we have $xy \in H$ and $x^{-1} \in H$. We denote $H \leq G$.
    \end{defn}

\end{document}
	    
	    \chapter{Topology}
	        %\documentclass{report}

\begin{document}
    \section{Topological Space}
    
    \subsection{Topological Space}
    
    In analysis, we've dealt with functions in metric spaces and their properties. What we will do in this chapter is extend this notion to the spaces without metrics. But without metrics, our definition of open sets no longer makes sense. We need a new definition.
    
    Remember the theorem [\ref{thm_union_open}] stating that a union of open sets is open, and a finite intersection of open sets is also open? Well...
    
    \begin{defn}[Topological Space] \label{def_topological_space}
        A \emph{topological space} is a set $X$ together with a collection $\mathcal{T}$ of subsetes of $X$ such that \begin{enumerate}
            \item $\emptyset \in \mathcal{T}$ and $X \in \mathcal{T}$.
            \item A union of sets in $\mathcal{T}$ is also in $\mathcal{T}$.
            \item A finite intersection of sets in $\mathcal{T}$ is also in $\mathcal{T}$.
        \end{enumerate}
        
        $\mathcal{T}$ is a \emph{topology} on $X$, and the sets in $T$ are called \emph{open} sets. The complement of an open set is a \emph{closed} set.
    \end{defn}
    
    \begin{defn} \label{def_topological_space_example} \begin{itemize}
        \item []
        \item Given a set $X$, the power set $\mathcal{P}(X)$ is the \emph{discrete topology}. This space is called the \emph{discrete space}. The set $\{\emptyset, X\}$ is the \emph{indiscrete topology}.
        \item A subset is \emph{cofinite}, and \emph{cocountable}, if its complement is finite, and countable, respectively. The set of $\emptyset$, $X$, and all cofinite subsets of $X$, together forms the \emph{cofinite topology}. Replacing cofinite with cocountable, we get the \emph{cocountable topology}.
    \end{itemize}
    \end{defn}
    
    From now on, we will assume $X$ and $Y$ are topological spaces, unless stated otherwise.
    
    \begin{defn} \label{def_topology_points}
        Let $A \subseteq X$. \begin{itemize}
            \item A point $x \in A$ is an \emph{interior point} of $A$ if some open neighborhood of $x$ is contained in $A$. The set of all interior points of $A$ is the \emph{interior} of $A$, denoted $int(A)$.
            \item A point $x \in X$ is an \emph{adherent point} of $A$ if every open neighborhood of $x$ intersects $A$. The set of all adherent points of $A$ is the \emph{closure} of $A$, denoted $\bar{A}$.
            \item The \emph{boundary} of $A$ is $\partial A = \bar{A} \cap \bar{(X \backslash A)}$.
            \item $x \in X$ is a \emph{limit point} of $A$ if every open neighborhood of $x$ contains at least one point in $A$ different from $x$. The set of all limit points of $A$ is denoted $A'$.
            \item $x \in A$ is an \emph{isolated point} of $A$ if some open neighborhood of $x$ does not contain any point in $A$ different from $x$. The set of all isolated points of $A$ is denoted $A^\cdot$.
    \end{itemize} \end{defn}
    
    \subsection{Base}
    
    \subsection{Continuity and Convergence}
    
    \subsection{Subspaces}
    
    \section{Connected Spaces}
    
    \subsection{Connectedness}
    
    \subsection{Total Disconnectedness}
    
    \subsection{Path Connectedness}
    
    \section{Separation Axioms}
    
    \section{Countability Axioms}
    
    \section{Compact Spaces}
    
    \subsection{Compactness}
    
    \subsection{Other Types of Compactness}
    
    \subsection{Boundedness}
    
    \section{Metrization}
    
    \section{Sequence of Functions}
    
    \section{Paracompact Spaces}

\end{document}
	        
	    \chapter{Matroid}
	        %\documentclass{report}

\begin{document}
    \section{something something}

\end{document}
	   
	    \chapter{Lebesgue Integration}
	        %\documentclass{report}

\begin{document}
    \section{something something}

\end{document}

	\part{Applications to Computer Science}
	
		\chapter{Language Theory}
		    \input{./chapters/applications/language_theory.tex}
		
		\chapter{Theory of Computation}
		    %\documentclass{report}

\begin{document}
    \section{Computability}
    
    Turing Machine was already defined in [\ref{def_TM}], but let's write down the definition here for convenience:
    
    \begin{defn}[Turing Machine]
		A Turing machine consists of:
		\begin{itemize}
			\item A \emph{tape} divided into consecutive cells. Each cell contains a symbol from the tape alphabet, which contains a blank symbol and one or more other symbols. The tape is assumed to be infinitely long to the left; cells that have not been written before are assumed to be filled with the blank symbol.
			\item A \emph{head} which can read a single symbol on the tape at a time, and is able to move one(and only one at once) cell to the right or the left.
				\item A \emph{state register} which stores the state of the TM, starting from the starting state(defined below) and following the transition function's rule(also defined below).
		\end{itemize}
		Formally, a TM is a 7 tuple $(Q,\Sigma,\Gamma,\delta,q_0,q_{accept},q_{reject})$ where:
		\begin{itemize}
			\item $Q$ is the set of states;
			\item $\Gamma$ is the set of tape alphabet;
			\item $b \in \Gamma$ is the blank symbol, the only symbol allowed to occur infinitely often at any step of the computation;
			\item $\Sigma \subseteq \Gamma \textbackslash \{b\}$ is the set of input symbols, that is, the set of symbols allowed to appear in the initial tape contents;
			\item $q_0 \in Q$ is the starting state;
			\item $F \subseteq Q$ is the set of accepting states, and the initial tape contents is said to be accepted by $M$ if it eventually halts in a state from $F$;
			\item $\delta$ is a partial function called the transition function of $(Q \textbackslash F) \times \Gamma \rightarrow Q \times \Gamma \times \{L,R\}$, where $L$ and $R$ signifies left and right shifts of the tape. If $\delta$ is undefined on the current state and the current tape symbol, then the machine halts.
			\end{itemize}
			Using the components of TM and the formal definition, the Turing machine accepts iff it halts on the set of accepting states, and it rejects iff it halts on the set of rejecting states. It may loop infinitely, of which it neither accepts nor rejects the tape.
		\end{defn}
	
	Usually, the proofs involving Turing machines do not give a formal construction of the machines because it is an extremely tedious process. Instead the proof describes what the machine does. It would be intuitive to see that such a machine can indeed be constructed.
	
	\begin{defn}[Decision Problem] \label{def_problem} \begin{itemize}
	    \item []
	    \item A \emph{decision problem} is a subset of the set of natural numbers $\mathbb{N}$. We assume $0 \in \mathbb{N}$.
	    \item An \emph{input} is a natural number that will be written on the initial tape in binary.
	    \item A Turing machine $M$ \emph{accepts} an input $x$ if it accepts with the given input $x$. Similarly, $M$ \emph{rejects} $x$ if it rejects with the given input $x$.
	    \item A Turing machine $M$ \emph{decides} a decision problem $L$ if, for all $x \in \mathbb{N}$, $M$ accepts $x$ if $x \in L$ and $M$ rejects $x$ otherwise. In that case, $L$ is \emph{decidable}.
	    \item A Turing machine $M$ \emph{semi-decides} a decision problem $L$ if, for all $x \in \mathbb{N}$, $M$ accepts $x$ if $x \in L$ and $M$ runs infinitely otherwise. In that case, $L$ is \emph{semi-decidable}.
	\end{itemize} \end{defn}
	
	What if we want other types of inputs such as two natural numbers, rational numbers, ASCII strings, graphs, and so on? In that case, we can \emph{encode} them as natural numbers. For example, there are some easy-to-compute injection from $\mathbb{N}^2$ to $\mathbb{N}$, and we assume they are given as the encoded forms. But since there is no injection from $\mathbb{R}$ to $\mathbb{N}$, we cannot give real numbers as inputs.

    \section{Computational Complexity}
    (TODO: Write something about asymptotic notation here)
        
    \begin{defn}[Asymptotic notation] \label{def_bigo}
        Let $f$ and $g$ be two functions from $\mathbb{N}$ to $\mathbb{N}$. Then we say: \begin{itemize}
            \item $f=O(g)$ if there is a constant $c$ such that $f(n) \leq c \cdot g(n)$ for every sufficiently large $n$. That is, $n>N$ for some $N$.
            \item $f=\Omega(g)$ if $g=O(f)$.
            \item $f=\Theta(g)$ if $f=O(g)$ and $g=O(f)$.
            \item $f=o(g)$ if for every constant $c>0$, $f(n) < c \cdot g(n)$ for every sufficiently large $n$.
            \item $f=\omega(g)$ if $g=o(f)$.
        \end{itemize} 
    \end{defn}
        
    \begin{defn}[P, NP, EXP] \label{def_comp_p}
        \begin{itemize}
            \item []
            \item $\mathbf{P}$ is the set of boolean functions computable with a deterministic Turing machine in time $O(n^c)$ for some constant $c>0$.
            \item $\mathbf{NP}$ is the set of boolean functions computable with a non-deterministic Turing machine in time $O(n^c)$ for some constant $c>0$.
            \item $\mathbf{EXP}$ is the set of boolean functions computable with a deterministic Turing machine in time $O(2^{n^c})$ for some constant $c>0$.
        \end{itemize}
    \end{defn}
        
    \begin{thm} \label{thm_pnpexp}
        $\mathbf{P} \subseteq \mathbf{NP} \subseteq \mathbf{EXP}$.
    \end{thm}
        
    \begin{proof}
        .
    \end{proof}
    
    \section {Reduction}
    Is there a polynomial-time algorithm for a given decision problem? Computer scientists are interested in this question because if there is one, it is usually a small-degree polynomial like $O(n^2)$ or $O(n^5)$. Some problems have a special property that if the problem has a polynomial-time algorithm, then several other problems do.
        
    \begin{defn}[Polynomial-time Karp reduction] \label{def_poly_karp}
        A problem $A \subseteq \strs$ is \emph{polynomial-time Karp reducible} to $B \subseteq \strs$, denoted $A \leq_p B$, if there is a polynomial-time computable function $f: \strs \rightarrow \strs$ such that for every $x \in \strs$, $x \in A$ iff $f(x) \in B$.
    \end{defn}
        
    The intuitive meaning is that a problem of $A$ can be "reduced" to a problem of $B$, and if we can solve $B$ in polynomial-time, then we can solve $A$ in polynomial-time too.
        
    \begin{defn}[NP-complete] \label{def_npc}
        A problem $A$ is \emph{NP-hard} if every problem in $\mathbf{NP}$ is polynomial-time reducible to $A$, and \emph{NP-complete} if $A$ is NP-hard and NP.
    \end{defn}
        
    \begin{thm} \label{thm_leqp_transitive}
        \begin{enumerate}
            \item[]
            \item If $A \leq_p B$ and $B \leq_p C$, then $A \leq_p C$.
            \item An NP-complete problem $A$ is in $\mathbf{P}$ iff $\mathbf{P}=\mathbf{NP}$.
            \item If $A \leq_p B$ and $A$ is NP-hard, then $B$ is NP-hard.
        \end{enumerate}
        %If $A \leq_p B$ and $B \leq_p C$, then $A \leq_p C$. Also, an NP-complete problem $A$ is in $\mathbf{P}$ iff $\mathbf{P}=\mathbf{NP}$.
    \end{thm}
        
    \begin{proof}
        (1) Let $f$ be a reduction from $A$ to $B$ with polynomial time $p(n)$, and $g$ from $B$ to $C$ with $q(n)$. Then $g \circ f$ is a reduction from $A$ to $C$ with polynomial time $q(p(n))$.
            
        (2) Suppose $A$ is NP-complete and in $\mathbf{P}$. Then any problem $B$ in $\mathbf{NP}$ can be polynomial-time reduced to $A$, so transitivity implies that $B$ is polynomial-time computable. The converse is trivial.
            
        (3) Any problem $C$ in $\mathbf{NP}$ can be polynomial-time reduced to $A$. Transitivity implies that $C$ can be polynomial-time reduced to $B$.
    \end{proof}
        
    Now the obvious question is, does such a strong problem actually exist? The answer is yes, and a lot of important problems are NP-complete.
        
    (TODO: SAT)
        
    Having proven that SAT is NP-hard, more problems can be proven NP-hard if we can reduce SAT to those problems in polynomial-time. Here are only a tiny fraction of the NP-complete problems:
        
    \begin{defn}[NP-complete problems] \label{def_npc_examples} \begin{itemize}
        \item[]
        \item The \emph{3-SAT problem} is a SAT problem where each clause contains exactly 3 variables.
        \item Given a graph $G$ and an integer $0 \leq k \leq |V(G)|$, the \emph{clique problem} asks whether there is a complete induced subgraph of $G$ with size at least $k$.
        \item The \emph{independent set problem} asks whether there is a subset $S$ of $V(G)$ with size at least $k$ such that no two vertices in $S$ are adjacent, and 0 otherwise.
        \item The \emph{vertex cover problem} asks whether there is a subset $S$ of $V(G)$ with size at most $k$ such that each edge is adjacent to at least one vertex in $S$.
        \item The \emph{chromatic number problem} asks whether $G$ is 3-colorable.
        \item Given a set $S$ of integers and an integer $k$, the \emph{subset sum problem} asks whether there is a subset of $S$ whose sum of elements equals $k$.
        \item Given an $n \times m$ matrix $A$ and an $n \times 1$ matrix $b$ of integers, the \emph{integer programming problem} asks whether there is an $m \times 1$ matrix $x$ of integers such that each element of $Ax+b$ is non-negative.
    \end{itemize} \end{defn}
        
    \begin{thm} \label{thm_npc_examples}
        All problems in [\ref{def_npc_examples}] are NP-complete.
    \end{thm}
        
    \begin{proof}
    Clearly all problems described are NP. We will only show that they are all NP-hard.
        
    If we can reduce SAT to 3-SAT in polynomial time, then [\ref{thm_leqp_transitive}] will show that 3-SAT is NP-hard. To do this, note that \begin{itemize}
        \item $x$ is equivalent to $x \vee x \vee x$,
        \item $x_1 \vee x_2$ is equivalent to $x_1 \vee x_2 \vee x_2$,
        \item $x_1 \vee \cdots \vee x_n$ is equivalent to $(x_1 \vee x_2 \vee y_1) \wedge (\neg y_1 \vee x_3 \vee y_2) \wedge \cdots (\neg y_{n-4} \vee x_{n-2} \vee y_{n-3}) \wedge (\neg y_{n-3} \vee x_{n-1} \vee x_{n})$, where $n \geq 4$ and $y_1,\cdots,y_{n-3}$ are new variables unused in the original SAT formula.
    \end{itemize}
        
    Next, we reduce 3-SAT to a clique problem. (TODO)
        
    $G$ has a clique of size $k$ iff $\bar{G}$ has an independent set of size $k$. This shows that clique and independent set are polynomial-time reducible to each other.
        
    $G$ has an independent set of size $k$ iff $G$ has a vertex cover of size $|V(G)| - k$, by taking the complement of the independent set. Therefore independent set and vertex cover are polynomial-time reducible to each other.
        
    We reduce 3-SAT to a chromatic number problem. (TODO)
        
    We reduce 3-SAT to a subset sum problem. (TODO)
        
    Finally, we reduce 3-SAT to an integer programming problem. Given a 3-SAT formula with $n$ variables, set $0 \leq x_i \leq 1$ for $i=1,\cdots,n$, and convert the clause $(x_a \vee x_b \vee x_c)$ into $x_a + x_b + x_c \geq 1$. If the clause contains $\neg x_a$, convert it to $1-x_a$. This system of inequalities can easily be converted to the matrix form.
        
    \end{proof}
\end{document}
		
		\chapter{Graph Theory}
		    %\documentclass{report}

\begin{document}
    \section{Basic Graph Definitions}
        
        \begin{defn}[Graph] \label{def_graph}
            A \emph{graph} $G$ is represented by a pair of sets $(V(G), E(G))$, and a relation $\sim _G \subseteq V(G) \times E(G)$ such that for each $e \in E(G)$, there are exactly one or two $v \in V(G)$ such that $v \sim _G e$. An element of $V(G)$ is a \emph{vertex}, and an element of $E(G)$ is an \emph{edge}. If $v \sim _G e$, we say $v$ is \emph{incident} with $e$, and $v$ is an \emph{end} of $e$.
        \end{defn}
        
        \begin{figure}[h] \centering \begin{tikzpicture}
            \draw[fill=black] (0,0) circle (3pt);
            \draw[fill=black] (0,2) circle (3pt);
            \draw[fill=black] (1,1) circle (3pt);
            \draw[fill=black] (3,1) circle (3pt);
            \draw[fill=black] (4,0) circle (3pt);
            \draw[fill=black] (4,2) circle (3pt);
            \draw[thick] (1,1) -- (0,2) -- (0,0) -- (1,1) -- (3,1) -- (4,0) -- (4,2) -- (3,1);
        \end{tikzpicture}
        \caption{A graph with 6 vertices and 7 edges.} \label{fig-graph_example}
        \end{figure}
        
        From now on, we will skip $(G)$ and just write $V$ and $E$ if the context is obvious. Similarly we will skip $_G$ and just write $\sim$. Also, for simple graphs, we may write an edge as $vw$ where $v$ and $w$ are the ends of the edge.
        
        \begin{defn} \label{def_graph_terms} \begin{itemize}
            \item[]
            \item A vertex $v$ is \emph{adjacent} to another vertex $w$ if there is an edge $e$ such that $v \sim e$ and $w \sim e$. We also say that $v$ is a \emph{neighbor} of $w$.
            \item A \emph{loop} is an edge with exactly one end.
            \item Two edges $e_1$ and $e_2$ are \emph{parallel} if $e_1 \neq e_2$ and the set of ends of $e_1$ equals that of $e_2$.
            \item A graph $G$ is \emph{simple} if it has no loops or parallel edges.
            \item Two graphs $G$ and $H$ are \emph{isomorphic} if there are two bijections $f_V : V(G) \to V(H)$ and $f_E : E(G) \to E(H)$ such that for all $v \in V(G)$ and $e \in E(G)$, $v \sim_G e$ iff $f_V(v) \sim_H f_E(e)$.
        \end{itemize} \end{defn}
        
        Note that some texts might use a different definition of graphs. One common definition is that $E(G)$ is a set of two-element subsets of $V(G)$. With this definition, our definition of a simple graph is just called a graph, and our definition of a graph is called a multigraph.
        
        \begin{defn}[Subgraph] \label{def_subgraph} \begin{itemize}
            \item[]
            \item A graph $G$ is a \emph{subgraph} of a graph $H$ if $V(G) \subseteq V(H)$, $E(G) \subseteq E(H)$, with the same incidence relation, i.e. the set of ends of any edge $e$ in $G$ equals that of $e$ in $H$.
            \item For $e \in E$, $G \backslash e$ is $(V(G), E(G) \backslash \{e\})$ with the same incidence relation.
            \item For $v \in V$, $G \backslash v$ is $(V(G) \backslash \{v\}, E')$, where $E'$ is the set of edges in $G$ not incident with $v$, with the same incidence relation.
            \item A subgraph $H$ of $G$ is \emph{spanning} if $V(H) = V(G)$.
            \item A subgraph $H$ of $G$ is \emph{induced} if $E(H)$ equals the set of edges in $G$ whose set of ends is contained in $V(H)$. We say $H$ is \emph{induced by} $V(H)$.
            \item For $X \subseteq V$, $G[X]$ is a subgraph of $G$ induced by $X$.
        \end{itemize} \end{defn}
        
        \begin{defn} \label{def_graph_example} \begin{itemize}
            \item[]
            \item A \emph{complete graph} with $n$ vertices, denoted $K_n$, is a simple graph in which for any pair of different vertices there is an edge connecting them.
            \item A \emph{cycle graph} with $n$ vertices, denoted $C_n$, is a simple graph whose edge set is $\{v_1v_2, \cdots, v_{n-1}v_n, v_nv_1\}$, where $V = \{v_1, \cdots, v_n\}$.
            \item A graph $G$ is \emph{bipartite} if $V$ can be partitioned into non-empty subsets $A$ and $B$ such that no edges connect two vertices in $A$ or two vertices in $B$.
            \item A \emph{complete bipartite graph} with $n+m$ vertices, denoted $K_{n,m}$, is a simple bipartite graph with $|A|=n$, $|B|=m$ in which for any vertex in $A$ and in $B$, there is an edge connecting them.
            \item For a simple graph $G$, the \emph{complement} $\bar{G}$ of $G$ is a simple graph on $V(G)$ such that any two different vertices $v$ and $w$ are adjacent in $G$ iff they are not adjacent in $\bar{G}$.
        \end{itemize} \end{defn}
        
        \begin{figure}[h] \centering \begin{tikzpicture}
            % K3
            \draw[fill=black] (0,0) circle (3pt);
            \draw[fill=black] (0,1.5) circle (3pt);
            \draw[fill=black] (1.3,0.75) circle (3pt);
            \draw[thick] (0,0) -- (0,1.5) -- (1.3,0.75) -- (0,0);
            % K4
            \draw[fill=black] (2,0) circle (3pt);
            \draw[fill=black] (2,1.5) circle (3pt);
            \draw[fill=black] (3.5,0) circle (3pt);
            \draw[fill=black] (3.5,1.5) circle (3pt);
            \draw[thick] (2,0) -- (2,1.5) -- (3.5,1.5) -- (3.5,0) -- (2,0) -- (3.5,1.5) -- (3.5,0) -- (2,1.5);
            % C3
            \draw[fill=black] (4,0) circle (3pt);
            \draw[fill=black] (4,1.5) circle (3pt);
            \draw[fill=black] (5.3,0.75) circle (3pt);
            \draw[thick] (4,0) -- (4,1.5) -- (5.3,0.75) -- (4,0);
            % C4
            \draw[fill=black] (6,0) circle (3pt);
            \draw[fill=black] (6,1.5) circle (3pt);
            \draw[fill=black] (7.5,0) circle (3pt);
            \draw[fill=black] (7.5,1.5) circle (3pt);
            \draw[thick] (6,0) -- (6,1.5) -- (7.5,1.5) -- (7.5,0) -- (6,0);
            % K23
            \draw[fill=black] (8,0.3) circle (3pt);
            \draw[fill=black] (8,1.2) circle (3pt);
            \draw[fill=black] (9.5,0) circle (3pt);
            \draw[fill=black] (9.5,0.75) circle (3pt);
            \draw[fill=black] (9.5,1.5) circle (3pt);
            \draw[thick] (8,0.3) -- (9.5,0) -- (8,1.2) -- (9.5,0.75) -- (8,0.3) -- (9.5,1.5) -- (8,1.2);
            % C4 complement
            \draw[fill=black] (10,0) circle(3pt);
            \draw[fill=black] (10,1.5) circle(3pt);
            \draw[fill=black] (11.5,1.5) circle(3pt);
            \draw[fill=black] (11.5,0) circle(3pt);
            \draw[thick] (10,0) -- (11.5,1.5);
            \draw[thick] (10,1.5) -- (11.5,0);
        \end{tikzpicture}
        \caption{From left to right: $K_3$, $K_4$, $C_3$, $C_4$, $K_{2,3},$ and $\bar{C_4}$. Note that $K_3$ is isomorphic to $C_3$.} \label{fig-K3_K4}
        \end{figure}
        
        \begin{defn} \label{def_graph_path} \begin{itemize}
            \item[]
            \item A \emph{walk} from $v \in V$ and $w \in V$ is an alternating sequence $v_0 e_1 v_1 e_2 \cdots e_k v_k$ of vertices and edges such that $v_0 = v$, $v_k = w$, and the set of ends of $e_i$ equals $\{v_{i-1}, v_i\}$. $k$ is the \emph{length} of the walk.
            \item A \emph{trail} is a walk with distinct edges.
            \item A \emph{closed walk} is a walk with $v=w$ and $k>0$.
            \item A \emph{circuit} is a trail that is also a closed walk.
            \item A \emph{path} is a walk with distinct vertices.
            \item A \emph{cycle} is a circuit with distinct $\{v_0, \cdots, v_{k-1}\}$.
        \end{itemize} \end{defn}
        
        \begin{defn}[Connectivity] \label{def_connected_graph} \begin{itemize}
            \item []
            \item A graph is \emph{connected} if for any two vertices in $V$ there is a path connecting them.
            \item A \emph{connected component} of a graph $G$ is $G[X]$ such that $G[X]$ is connected, and for any $Y \subseteq V$ such that $X \subsetneq Y$, $G[Y]$ is not connected.
        \end{itemize} \end{defn}
    
    \section{Degrees}
    
        \begin{defn}[Degree] \label{def_graph_degree}
            The \emph{degree} of $v \in V(G)$, denoted $deg_G(v)$, is the number of non-loop edges incident with $v$, plus two times the number of loops incident with $v$.
        \end{defn}
        
        Again, we might skip $G$ and write $deg(v)$. As we progress, it will be clear why it is convenient to count a loop twice.
        
        \begin{lemma}[Degree Sum Formula] \label{lem_degree_sum}
            $\sum_{v \in V}deg(v) = 2|E|$.
        \end{lemma}
        
        \begin{proof}
            Induction on $|E|$, with the trivial base case $|E| = 0$. Suppose $|E| > 0$. Let $\sum_{v \in V}deg(v) = A$ and $2|E| = B$. Take any edge $e$, and the induction with $G\backslash e$ shows that $A-2 = B-2$. Therefore $A=B$.
        \end{proof}
        
        \begin{lemma}[Handshaking Lemma]
            A graph has an even number of odd-degree vertices.
        \end{lemma}
        
        \begin{proof}
            $2|E|$ is an even number. From [\ref{lem_degree_sum}], exactly even number of the terms $deg(v)$ must be odd.
        \end{proof}
        
        The degree sum formula is sometimes also called the handshaking lemma.
        
        \begin{defn}[Degree Sequence] \label{def_graph_score}
            The \emph{degree sequence} of a graph $G$, or the \emph{score} of $G$, is the sequence of degrees $(deg(v_1),\cdots,deg(v_{|V|}))$.
        \end{defn}
        
        Now, how can we figure out if a sequence is a degree sequence of some graph? The following theorem gives a simple $O(\sum d_i)$-time algorithm to answer the question:
        
        \begin{thm}[Havel-Hakimi Algorithm] \label{thm_havel_hakimi}
            Let $(d_1,\cdots,d_n)$ be a sequence of integers such that $0 \leq d_1 \leq \cdots \leq d_n$ and $n>1$. It is a degree sequence of some simple graph iff $(d_1,\cdots,d_{z-1},d_z-1,\cdots,d_{n-1}-1)$ is a degree sequence of some simple graph, where $z = n-d_n$.
        \end{thm}
        
        \begin{proof}
            ($\Leftarrow$) If $(d_1,\cdots,d_{z-1},d_z-1,\cdots,d_{n-1}-1)$ is a degree sequence of some simple graph, then we can make $(d_1,\cdots,d_n)$ by adding a vertex and connecting to the vertices with degrees $d_z-1,\cdots,d_{n-1}-1$.
            
            ($\Rightarrow$) Let $G$ be a simple graph such that $deg_G(v_i)=d_i$ for all $v_i \in V(G)$. We will construct a simple graph $H$ with $deg_H(u_i)=d_i$ for all $u_i \in V(H)$ such that $v_n$ is connected to $v_{n-d_n},\cdots,v_n-1$. Then the conclusion follows by taking $H-v_n$.
            
            If $d_n=n-1$, then simply take $H=G$. Otherwise, define $j(G)$ as the largest index $j$ such that $v_n$ is not adjacent to $v_j$. Among all graphs with $deg_H(u_i)=d_i$, take one graph with the smallest $j(H)$. (Note that such $H$ exists because at least one graph, namely $G$, satisfies the degree sequence condition.)
            
            Suppose $j=j(H) \geq n-d_n$. Then there is an index $i<j$ such that $u_n$ is adjacent to $u_i$. Since $deg_H(u_i) \leq deg_H(u_j)$, there is a vertex $w$ adjacent to $u_j$ but not to $u_i$. Now, consider a new graph $H'$ derived from $H$ by removing $u_iu_n$ and $u_ju_k$, and adding $u_ju_n$ and $u_iu_k$. Then $deg_{H'}(u_i)=d_i$ and $j(H')<j(H)$, contradicting the minimality of $H$. Therefore $j(H) = n-d_n$.
        \end{proof}
    
    \section{Trees}
        One of the important classes of graphs is a tree. There are many ways to define a tree. First we will state one definition, and then prove that other definitions are equivalent.
        \begin{defn}[Tree] \label{def_tree}
            A \emph{forest} is a simple graph without any cycle. A \emph{tree} is a connected forest. A \emph{leaf} of a forest is a vertex with degree 1.
        \end{defn}
        
        \begin{thm} \label{thm_tree}
            The following statements are equivalent for a simple graph $G$: \begin{enumerate}
                \item $G$ is a tree.
                \item For any two vertices $u$ and $v$ of $G$, there is exactly one path connecting them.
                \item $G$ is connected, and for any $e$ of $G$, $G \backslash e$ is disconnected.
                \item $G$ has no cycle, and for any two vertices $u$ and $v$ not having an edge between them, $G+uv$ has a cycle.
                \item $G$ is connected, and $|E| = |V|-1$.
            \end{enumerate}
        \end{thm}

        \begin{proof}
            TODO
            %(1 $\Rightarrow$ 4) $T$ has no cycles. For any $x,y \in V$, there is a path from $y$ to $x$. That path and $xy$ creates a cycle in $T+xy$.
            
            %(4 $\Rightarrow$ 3) For any $x,y \in V$, $T+xy$ has a cycle and that cycle must contain $xy$. Removing $xy$ from the cycle yields a path from $x$ to $y$. Therefore $T$ is connected. Now suppose $T \backslash e$ is connected for some $e \in E$, and $e$ connects $x$ and $y$. There is a path from $y$ to $x$ in $T \backslash e$, and that path and $e$ creates a cycle in $T$: contradiction.
            
            %(3 $Righatrrow$ 2) $G$ is connected.
        \end{proof}
        
        \begin{thm} \label{thm_tree_leaf}
            A tree with at least 2 vertices has at least 2 leaves.
        \end{thm}
        
        \begin{proof}
            From [\ref{thm_tree}](5) and [\ref{lem_degree_sum}], the sum of degrees is $2|V|-2$. None of the degrees are 0. Therefore at least 2 degrees must equal 1.
        \end{proof}
        
        How can we figure out if a sequence is a degree sequence of some tree? It turns out to be a lot simpler than [\ref{thm_havel_hakimi}] and basically anything that makes sense can be a degree sequence of a tree:
        
        \begin{thm}[Degree Sequence of a Tree] \label{thm_tree_degree}
            A sequence $(d_1,\cdots,d_n)$ is a degree sequence of some tree iff all $d_i$ are positive and $\sum d_i = 2n-2$.
        \end{thm}
        
        \begin{proof}
            ($\Rightarrow$) Clear from [\ref{thm_tree}](5) and [\ref{lem_degree_sum}].
            
            ($\Leftarrow$) Induction on $n$, with trivial base cases $n \leq 2$. Now suppose $n \geq 3$. There exists $i$ and $j$ such that $d_i=1$ and $d_j>1$; WLOG assume $i=1$ and $j=2$. From induction, $(d_2-1,d_3,\cdots,d_n)$ is a degree sequence of some tree. Take any vertex $v$ in the tree with degree $d_2-1$, and add a leaf adjacent to $v$. This constructs a tree with the degree sequence $(d_1,\cdots,d_n)$.
        \end{proof}
    
    \subsection{Spanning Trees}
        \begin{defn}[Spanning Subgraph] \label{def_spanning}
            A \emph{spanning subgraph} of a graph $G$ is a subgraph of $G$ such that its vertex set equals $V(G)$. A \emph{spanning tree} is a spanning graph that is a tree.
        \end{defn}
        
        \begin{thm} \label{thm_spanning_tree}
            A connected graph $G$ has a spanning tree.
        \end{thm}
        
        \begin{proof}
            Let $m = |E|$, and label the edges as $e_0$, $\cdots$, $e_m$, arbitrarily. Define the subsets $E_0$, $\cdots$, $E_m$ of $E$, as \begin{displaymath} \begin{cases}
                E_0 = \varnothing \\
                E_i = E_{i-1} \cup \{e_i\} & \text{if the spanning subgraph of $G$} \\
                & \text{with $E = E_{i-1} \cup \{e_i\}$ has no cycle} \\
                E_i = E_{i-1} & \text{otherwise.}
            \end{cases} \end{displaymath}
            Let $H$ be the spanning subgraph of $G$ with $E = E_m$. Clearly, $H$ has no cycle. If $e_i \notin E_m$ and $H+e_i$ has no cycle, then $E_i$ would contain $e_i$, contradiction. From [\ref{thm_tree}], $H$ is a tree.
        \end{proof}
        
        (TODO: minimum spanning tree)
        
        There are other minimum spanning tree algorithms like Prim's algorithm or Borůvka's algorithm.
    
    \section{Planar Graphs}
        \begin{defn}[Planar Graph] \label{def_planar_graph}
            A \emph{plane graph} is a graph $G$ where: \begin{itemize}
                \item $V \subseteq \mathbb{R}^2$;
                \item every edge is an arc between two endpoints;
                \item the interior of each edge contains no vertex and no point of any other edge.
            \end{itemize}
            The connected components of $\mathbb{R}^2 \backslash G$ are called \emph{faces} of $G$. Since $G$ is contained in a sufficiently large disc, exactly one face is unbounded; that face is called the \emph{outer face} of $G$. All other faces are called \emph{inner faces} of $G$. A graph $H$ is \emph{planar} if it is isomorphic to some plane graph.
        \end{defn}
        
        \begin{thm}[Euler's Formula] \label{thm_euler_vef}
            If $G$ is a connected plane graph, and the number of faces of $G$ is $F$, then \begin{displaymath}
                |V| - |E| + F = 2.
            \end{displaymath}
        \end{thm}
        
        \begin{proof}
            Induction on $|E|$. The base case is when $G$ has no edges, one vertex, and one face; the formula clearly holds.
            
            Pick any edge $e$. If $e$ is a loop, removing it reduces $|E|$ and $F$ by one. Otherwise, contracting it reduces $|V|$ and $|E|$ by one. Either way the result follows by induction.
        \end{proof}
        
        \begin{thm} \label{thm_planar_ve}
            If $G$ is simple and planar, and $|V| \geq 3$, then $|E| \leq 3|V|-6$. If in addition $G$ has no triangles (i.e. $K_3$ as a subgraph), then $|E| \leq 2|V|-4$.
        \end{thm}
        
        \begin{proof}
            Count the number $N$ of pairs $(f, e)$ where the face $f$ and the edge $e$ are incident. For each face, there are at least 3 edges incident to it, for otherwise there would be parallel edges or loops. Therefore $N \geq 3F$. On the other hand, each edge is incident to exactly two faces, so $N = 2|E|$. This gives $3F \leq 2|E|$. From [\ref{thm_euler_vef}], $3F = 6-3|V|+3|E| \leq 2|E|$, and the first result follows.
            
            The second result can be proved in the exactly same way, using $N \geq 4F$.
        \end{proof}
        
        \begin{coro} \label{cor_k5_k33}
            $K_5$ and $K_{3,3}$ are not planar.
        \end{coro}
        
        \begin{proof}
            $K_5$ has 5 vertices and 10 edges. $K_{3,3}$ has no triangles, 6 vertices, and 9 edges. The result follows from [\ref{thm_planar_ve}].
        \end{proof}
        
        TODO: add a figure of K5 and K33.
        
        It clearly follows that any subdivision of $K_5$ or $K_{3,3}$ are not planar. Surprisingly, those two graphs are the only graphs that ``need to be checked" to determine if a given graph is planar. The proof requires several more lemmas and theorems, so we have moved the proof to the appendix.
        
        \begin{thm}[Kuratowski's Theorem] \label{thm_kuratowski}
            A graph $G$ is planar if and only if it does not have $K_5$ or $K_{3,3}$ as a topological minor.
        \end{thm}
    
    \section{Coloring}
        \begin{defn}[Coloring] \label{def_coloring}
            A \emph{$k$-coloring} of a graph $G$ is a function $c: V(G) \to \{1,2,\cdots,k\}$ such that if $u$ and $v$ are adjacent vertices, then $c(u) \neq c(v)$. $G$ is \emph{$k$-colorable} if there is a $k$-coloring of $G$. The \emph{chromatic number} $\chi(G)$ of $G$ is the smallest integer $k$ such that $G$ is $k$-colorable.
        \end{defn}
        
        Perhaps the most famous theorem about graph coloring is the four-color theorem. (TODO: write something) 
        
        \begin{thm}[Four-color Theorem] \label{thm_four_color}
            If $G$ is planar, then $\chi(G) \leq 4$.
        \end{thm}
        
        Unfortunately, the proof is too long and complicated to contain in the codex. We prove a weaker result:
        
        \begin{thm}[Five-color Theorem] \label{thm_five_color}
            If $G$ is planar, then $\chi(G) \leq 5$.
        \end{thm}
        
        \begin{proof}
            Induction on $|V|$. For $|V| \leq 5$, the theorem is trivial.
            
            From [\ref{thm_planar_ve}], $G$ has a vertex $v$ of degree at most 5. If $deg_G(v) < 5$, then inductively find a 5-coloring of $G-v$, and color $v$ by some color in $\{1,2,3,4,5\}$ not appearing in the neighbors of $v$. If $deg_G(v) = 5$ and not all colors are used in the neighbors of $v$, then the same argument applies.
            
    \begin{figure}[h] \centering \begin{tikzpicture}
        \draw[thick] (0,0) -- (-0.5,1) -- (0,0) -- (0.5,1) -- (0,0) -- (1,0) -- (0,0) -- (0,-1) -- (0,0) -- (-1,0);
        \draw[fill=white] (0,0) circle (5pt);
        \draw[fill=red] (-0.5,1) circle (5pt);
        \node[white] at (-0.5,1) {1};
        \draw[fill=blue] (0.5,1) circle (5pt);
        \node[white] at (0.5,1) {2};
        \draw[fill=red] (1,0) circle (5pt);
        \node[white] at (1,0) {1};
        \draw[fill=yellow] (0,-1) circle (5pt);
        \node at (0,-1) {4};
        \draw[fill=black] (-1,0) circle (5pt);
        \node[white] at (-1,0) {5};
        
        \draw[thick] (3,0) -- (2.5,1) -- (3,0) -- (3.5,1) -- (3,0) -- (4,0) -- (3,0) -- (3,-1) -- (3,0) -- (2,0);
        \draw[fill=green] (3,0) circle (5pt);
        \node at (3,0) {3};
        \draw[fill=red] (2.5,1) circle (5pt);
        \node[white] at (2.5,1) {1};
        \draw[fill=blue] (3.5,1) circle (5pt);
        \node[white] at (3.5,1) {2};
        \draw[fill=red] (4,0) circle (5pt);
        \node[white] at (4,0) {1};
        \draw[fill=yellow] (3,-1) circle (5pt);
        \node at (3,-1) {4};
        \draw[fill=black] (2,0) circle (5pt);
        \node[white] at (2,0) {5};
    \end{tikzpicture}
    \end{figure}
            
            Now suppose all 5 colors are used. Denote the neighbors of $v$ as $u_1$, $u_2$, $u_3$, $u_4$, $u_5$, in clockwise order. Without loss of generality, we will assume that $c(u_i)=i$.
            
            The main idea of the rest of the proof is that we want to change the color of one of the neighbors, say change $c(u_i)$ to $k$. This is impossible if $u_i$ has a neighbor of color $k$, in which case we want to also change the color of that neighbor, to $k'$. But then that neighbor might have yet another neighbor of color $k'$, and this continues to form a chain. Hence we introduce the \emph{Kempe chain}, named after Alfred Kempe.
            
            Let $V_{ij}$ be the set of vertices $w$ in $G$ such that there is a path from $u_i$ to $w$ consisting of vertices of color $i$ or $j$. Note that if we switch the colors of the vertices in $V_{ij}$ (i.e. change $i$ to $j$ and $j$ to $i$), and leave everything else the same, then the result is still a coloring.
            
            If $V_{13}$ does not contain $u_3$, then switch the colors of the vertices in $V_{13}$ and color $v$ by 1.
            
            (TODO: picture)
            
            Otherwise, $V_{24}$ does not contain $u_4$; switch the colors of the vertices in $V_{24}$ and color $v$ by 2. This gives a 5-coloring of $G$.
            
            (TODO: picture)
        \end{proof}
        
        Fun fact: In 1879, the Kempe chain method was used to ``prove" the four-color theorem by Alfred Kempe. No one noticed that this ``proof" had an error until eleven years later when Percy Heawood found the error. What we saw above is the modification of the incorrect proof to prove the weaker theorem. The correct proof of four-color theorem was completed in 1976 by Kenneth Appel and Wolfgang Haken.
        
        Here is his ``proof." Argue similarly as above with induction. If $deg_G(v) = 4$ and all 4 colors are used, then apply the Kempe chain method. Now suppose $deg_G(v) = 5$ and all 4 colors are used. Then one color is used exactly twice.
        
        There are two cases: the two neighbors with that color are next to each other in clockwise order, or they are not. The first case is easy, just use the Kempe chain method. The second case is where the fun starts.
        
        (TODO: picture. u5-u4-u1-u2-u3 clockwise; u1 and u5 has the same color. Cetner is noted v.)
        
        WLOG, $u_k$ has color $k$. For convenience, color 5 is the same as color 1.
        
        If $V_{42}$ does not contain $u_2$, then switch the colors of the vertices in $V_{25}$ and color $v$ by 4. Otherwise, if $V_{43}$ does not contain $u_3$, then switch and color $v$ by 4. Otherwise, $V_{13}$ does not contain $u_3$ and $V_{52}$ does not contain $u_2$. Switch each chain and color $v$ by 1.
        
        (TODO: second case picture.)
        
        Can you find a critical error in this argument? If you want to know, refer to the appendix.
    
\end{document}
		
		\chapter{Cryptosystem}
		    \input{./chapters/applications/cryptosystem.tex}
	
	\part{Appendix}
		\chapter{Appendix}
		    %\documentclass{report}

\begin{document}
    \section{Equivalent Statements for Invertible Matrices}\label{equiv_invert}
    		For $n \times n$ matrix $A$, the followings are equivalent:
        \begin{enumerate}
        	\item $A$ is invertible.
        	\item $A\bf{x}=\mathbb{O}$ only has the trivial solution.
        	\item The reduced row echelon form of $A$ is $I_n$.
        	\item $A$ can be represented as a product of elementary matrices.
        	\item $A\bf{x}=\bf{b}$ is consistent $\forall n \times 1$ matrix $\bf{b}$.
        	\item $A\bf{x}=\bf{b}$ has exactly one solution $\forall n \times 1$ matrix $\bf{b}$.
        	\item $\det(A)\ne0$.
        	\item column vectors of $A$ are linearly independent.
        	\item row vectors of $A$ are linearly independent.
        	\item column vectors of $A$ span $\mathbb{R}^n$.
        	\item row vectors of $A$ span $\mathbb{R}^n$.
        	\item column vectors of $A$ form a basis for $\mathbb{R}^n$.
        	\item row vectors of $A$ form a basis for $\mathbb{R}^n$.
        	\item rank$(A)=n$
        	\item nullity$(A)=0$
        	\item $($Null$(A))^\perp=\mathbb{R}^n$
        	\item $($Row$(A))^\perp=\{\mathbb{O}\}$
        	\item range of $T_A$ is $\mathbb{R}^n$
        	\item $T_A$ is one-to-one.
        	\item $\lambda=0$ is not an eigenvalue of $A$.
        	\item $A^TA$ is invertible.
        \end{enumerate}
        
    \section{Cook-Levin Theorem}
        In this section
    
    \section{Kuratowski Theorem}
        In this section we prove [\ref{thm_kuratowski}].
        
        %(TODO: complete this, based on http://www.cs.xu.edu/~otero/math330/kuratowski.html http://www.cs.rpi.edu/~goldberg/14-GT/19-kurat.pdf and my lecture note)
    
    \subsection{The Preparation}
    
        First, we show that a planar graph can be drawn so that an arbitrary vertex or an edge is incident to the outer face.
    
        \begin{lemma} \label{lem_stereographic}
            If $G$ is planar and $v \in V(G)$, then there is a planar embedding of $G$ such that $v$ is on the boundary of the outer face. The same can be done for $e \in E(G)$.
        \end{lemma}
        
        \begin{proof}
            We use the \emph{stereographic projection}. In $\mathbb{R}^3$, let $z=-1$ be the plane $P$ and $x^2+y^2+z^2=1$ be the sphere $S$. $(0,0,1)$ is the ``north pole'' of $S$. Define the projection $\rho: S \backslash \{(0,0,1)\} \rightarrow P$ as follows: given $(x,y,z)$ on $S$ which is not the north pole, draw a straight line through $(0,0,1)$ and $(x,y,z)$. There is a unique intersection of this line with $P$, denoted as $(X,Y,-1)$. Then $\rho(x,y,z)=(X,Y,-1)$. Clearly $\rho$ is bijective.
            
            Given an embedding of a planar graph $G$ on $P$, $\rho^{-1}$ gives an embedding of $G$ on $S$. Rotate the embedding so that a face incident to $v$ or $e$ contains the north pole. $\rho$ gives an embedding of $G$ on $P$ such that the face is the outer face.
        \end{proof}
        
        Next, we introduce the notion of connectivity. Although connectivity is a crucial part of graph theory, we didn't put this into the main part of the codex because of the length concerns.
    
        \begin{defn}[Connectivity] \label{def_connectivity}
            A graph $G$ is \emph{$k$-connected} if $|V| > k$ and, for every $S \subset V$ with $|S| < k$, $G \backslash S$ is connected.
        \end{defn}
        
        \begin{thm} \label{thm_3conn}
            If $G$ is 3-connected with $|V(G)| \geq 5$, then there is an edge $e$ such that $G/e$ is 3-connected.
        \end{thm}
        
        \begin{proof}
            Let $e=xy$ and suppose $G/e$ is not 3-connected. Then $G/e$ has a cut set $\{v, z\}$. Since $G$ is 3-connected, this set has a vertex, say $v$, which is the new vertex made by contracting $e$. That is, $\{x, y, z\}$ is a cut set of $G$.
            
            Suppose that for every $e$, $G/e$ is not 3-connected, so to every $e$ corresponds a vertex $z_e$. Among all edges, take $e=xy$ and $z_e$ such that $G-x-y-z$ has the largest component $C$, and denote another component as $D$. Each of $x,y,z$ has neighbors in $C$ and in $D$ since $G$ is 3-connected. Take a neighbor $u$ of $z$ in $D$ and let $v=z_{zu}$.
            
            If $v \in V(C) \cup \{x,y\}$, then $G-z-v$ is disconnected, contradicting the connectivity of $G$. Otherwise, $G-z-u-v$ has a component that contains all vertices in $C$ and $x$ and $y$ in addition, contradicting the choice of $C$.
            
            (TODO: picture)
        \end{proof}
        
        Then, we show the connection between minors and topological minors.
        
        \begin{lemma} \label{lem_minor_K33}
            $K_{3,3}$ is a topological minor of $G$ iff $K_{3,3}$ is a minor of $G$.
        \end{lemma}
        
        \begin{proof}
            A topological minor of $G$ is also a minor of $G$. We just need to prove the other direction of the lemma.
            
            
        \end{proof}
        
        \begin{lemma} \label{lem_minor_K5}
            If $K_5$ is a minor of $G$, then $K_{3,3}$ or $K_5$ is a topological minor of $G$.
        \end{lemma}
        
        \begin{proof}
            .
        \end{proof}
    
    \subsection{The Proof}
    
        The last step is closely related to the Kuratowski's theorem.
    
        \begin{defn}[Convex Embedding] \label{def_convex_embed}
            A \emph{convex embedding} of a planar graph $G$ is a plane graph in which all edges are straight line segments and all face boundaries are convex polygons.
        \end{defn}
        
        \begin{lemma} \label{lem_convex_embed}
            If $G$ is simple, 3-connected, and has no $K_5$ or $K_{3,3}$ as a minor, then $G$ has a convex embedding on a plane, with no three vertices on a line.
        \end{lemma}
        
        \begin{proof}
            TODO
        \end{proof}
        
        We are finally ready to prove the Kuratowski's theorem. For convenience, we will restate the theorem:
        
        \begin{quote}
            A graph $G$ is planar if and only if it does not have $K_5$ or $K_{3,3}$ as a topological minor.
        \end{quote}
        
        \begin{proof}
            Induction on $|V|$, with trivial base case $|V| \leq 4$.
            
            If $G$ is disconnected, from induction there is a planar embedding of each component. Since each embedding is bounded by a finite disc, their union can be drawn on a plane.
            
            If $G$ is connected but not 2-connected, then take a cut-vertex $v$. Let $G_1$, $\cdots$, $G_n$ be the connected components of $G-v$, and $H_i$ be the subgraph induced by $V(G_i) \cup \{v\}$. Take an embedding of each $H_i$ such that $v$ is in the outer face [\ref{lem_stereographic}] and squeeze it into an angle $< 2\pi / n$ at the vertex $v$. Joining those embeddings together forms an embedding of $G$.
            
            If $G$ is 2-connected but not 3-connected, TODO
            
            If $G$ is 3-connected, the conclusion immediately follows from [\ref{lem_convex_embed}].
        \end{proof}
    
    \section{What's Wrong With Kempe's Proof?}
        
        Kempe argued that switching $V_{13}$ and $V_{52}$ allows $v$ to be colored by 1, but consider the following graph:
        
        (TODO: counterexample picture)
        
        Both chains cannot be switched because then the vertices $a$ and $b$ would have the same color!
        
        In this graph, such a problem could be avoided by deliberately changing the order of vertices to be selected for induction. However, there are graphs on which such a workaround is not possible. The following is the smallest counterexample possible, and is called the Soifer graph:
        
        (TODO: Soifer graph)
    
\end{document}

\end{document}